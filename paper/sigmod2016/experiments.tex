\section{Experimental Evaluation}
\label{sec:exp}

{\bf Experimental environment.} All experiments in this section are
conducted on an 8-node cluster.  \reminder{More details here:
  parameters of each slave instance, and that each instance is on a
  separate node.  Stand-alone Spark.}

All experiments are executed with a cold start, with
\insql{materialize} as the final statement.  Each experiment is
conducted 5 times, we report the average running time and some measure
of result variability. \reminder{how exactly do we measure
  variability?}

{\bf Data and evaluation methods.}  We evaluate performance of our
framework on three real open-source datasets.  

\begin{enumerate}

\item DBLP\footnote{\url{dblp.uni-trier.de/xml}} contains co-authorship
  information from 1936 through 2015, with over 1.5 million author
  nodes and over 6 million undirected co-authorship edges.  

\item nGrams\footnote{\url{storage.googleapis.com/books/ngrams/books/datasetsv2.html}}
  contains word co-occurrence information from 1520 through 2008, with
  over 1.5 million word nodes and over 65 million undirected
  co-occurrence edges.

\item
  DELIS\footnote{\url{law.di.unimi.it/webdata/uk-union-2006-06-2007-05}}
  contains monthly snapshots, from 05/2006 through 05/2007, of a
  portion of the Web graph focusing on the .uk domains, with a total
  of over 133 million nodes and over 5.5 billion directed
  edges~\cite{BSVLTAG}.

\end{enumerate}

Our evaluation focuses on efficiency of implementation. Experiments
are conducted on an in-house 8-node Hadoop cluster built with Open
Stack. \reminder{Additional details about nodes, distribution.}

{\bf Outline.}

\reminder{Vera, please write each of these out as \ql queries.  Also
  write what we are keeping fixed in each experiment, and what we are
  varying.}

\begin{enumerate}

\item \insql{TSelect}, to understand the number of
  partitions with which to load for SPG, MG, MGC and OG, as a function
  of data size.

\item \insql{TSelect} + transform (e.g., $a * a$, where $a$ is a
  scalar vertex attribute), to understand the overhead of computing a
  simple transformation operation, with the best settings in
  Experiment 1.  Probably no need for a full experiment,
  let's get a few data points to see whether the overhead is
  noticeable.

\item \insql{TSelect} with \insql{TGroup}, with
  \insql{Any} and \insql{All}.  We will run this experiment for each
  SPG, MG, MGC and OG, with the best setting as determined in
  Experiment 1.  We will then see whether re-partitioning improves
  performance, and will try different partitioning strategies.

\item \insql{TAnd} and \insql{All}, \insql{TOr} with \insql{Any}.  The
  same experiment as Experiment 3.

\item \insql{TSelect} with \insql{pagerank()}. The same
  experiment as Experiment 3.

\item \insql{pagerank()} with \insql{trend()}.  Run only for the best
  data structure of Experiment 5.

\item \insql{TSelect} with \insql{degree()}, only if we have time. The
  same experiment as Experiment 3.

\end{enumerate}
