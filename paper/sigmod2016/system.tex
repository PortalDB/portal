\section{System}
\label{sec:sys}

Our \ql system implementation builds on GraphX, an Apache Spark
library, as depicted in Figure~\ref{fig:arch}.  Green boxes indicate
built-in components while blue are those we added for \ql.  We
selected Apache Spark because it is a popular open-source system, and
because of its in-memory processing approach.  All language operators
are available through the public API of the \ql library, and may be
used like any other library in an Apache Spark application.

The \ql system includes an interactive shell for exporatory data
analysis and a query parser.  A \ql query is rewritten into a sequence
of operators, a particular data structure and execution method are
selected (Section~\ref{sec:sys:optimization}), and the query is
executed.  The evolving graph snapshots are read from a distributed
file system and processed by Workers, with the tasks assigned and
managed by the runtime.  We implement a variety of \tg representations
(Section~\ref{sec:sys:datastructs}) and partitioning strategies
(Section~\ref{sec:sys:partition}).  We will experimentally compare
preformance of these representations and partitioning strageties on
two real evolving graphs in Section~\ref{sec:exp}.

\begin{figure}[t!]
\begin{center}
\includegraphics[height=1.4in]{figs/architecture.pdf}
\caption{\ql system architecture.}
\label{fig:arch}
\end{center}
\vspace{-0.5cm}
\end{figure}

\subsection{Data Representation}
\label{sec:sys:datastructs}

We developed several in-memory representations of evolving graphs to
explore the tradeoffs of compactness, parallelism, and support of
different query operators. \eat{ The data structures represent a
  continuum of replication, from the SnapshotGraph to OneGraph and are
  described here in more detail.}

{\bf SnapshotGraph (SG).} The simplest way to represent an evolving
graph is by representing each snapshot individually, a direct
translation of our logical data model.  We call this data structure
SnapshotGraph, or SG for short. An example of an SG is depicted in
Figure~\ref{fig:sgp}.  SG is a collection of snapshots, where vertices
and edges store the attribute values for the specific time interval.
A \insql{TSelect} operation on this representation is a slice of the
snapshot sequence, while \insql{TGroup} and temporal joins
(\insql{TAnd} and \insql{TOr}) require a group by key within each
aggregate set of vertices and edges.

While the SG representation is simple, it is not compact, considering
that in many real-world evolving graphs there is a 80\% or larger
simmilarity between consecutive
snapshots~\cite{DBLP:journals/tos/MiaoHLWYZPCC15}.  In a distributed
architecture, however, this data structure provides some benefits as
operations on it can be easily parallelized, by assigning different
snapshots to different workers, or even partitioning a snapshot across
workers.  \eat{with improved temporal locality for snapshot-based
  analytics. }\eat{SG edges can be partitioned using temporal or
  structural criteria.  Furthermore, due to Spark's lazy evaluation,
  operations such as \insql{TSelect} are very efficient, since only
  those snapshots involved in the operation are loaded.  While other
  data structures do not benefit from this feature, push selection in
  query optimization can compensate equally well.}

\begin{figure}[t!]
\includegraphics[width=3.2in]{figs/sgp.pdf}
\caption{SG representation of T1 from Figure~\ref{fig:tg}.}
\label{fig:sgp}
\end{figure}

{\bf MultiGraph (MG).}  To take advantage of high similarity between
snapshots, we developed another data structure called MultiGraph, or
MG for short (Figure~\ref{fig:mg}).  MG stores the evolving graph as a
single graph, with one vertex for all time periods, but with one edge
per period where it exists.  Because our goal is to represent both
topoligical and attribute information, we need to store not only
vertex presence or absence (which can be easily accomplished by an
existence string, like in~\cite{Kan2009}, or by bit sets), but also
the values of the vertex attribute at each time period.  MG vertex
attribute, thus, is a map of time indices that represent intervals to
corresponding values.  Edge attributes are tuples of the time index
and the corresponding value.  Vertices typically change less
frequently than edges, so the space savings on storing each vertex
once are about 80\% in our experimental data sets.  Some of these
savings, however, are taken up by the storage of a more complex map
data structure for attribute values, compared to a simple single
attribute like in the SG.  \eat{Partition of the MG edges can be
  either temporal or structural, which lead to different rates of
  vertex replication between partitions.}

\begin{figure}[t!]
\includegraphics[width=3.2in]{figs/mg.pdf}
\caption{MG represetation of T1 from Figure~\ref{fig:tg}.}
\label{fig:mg}
\end{figure}

The implementation of some of the \ql operations in MG is more complex
than in SG.  \insql{TSelect} is a subgraph operation that operates on
all vertices and edges.  \insql{TGroup} on vertices is a combination
of transform and filter operations since the vertices are already
aggregated across the whole time period, but the edges, like in SG,
are grouped by key within their respective sets.  Implementation of
snapshot analytics like PageRank is done in batch mode, similar
to~\cite{DBLP:journals/tos/MiaoHLWYZPCC15}, by computing the values
and sending messages between vertices for all time periods at once.
\eat{This data structure should also be more amenable to cross-time
analytics and pattern mining, which we intend to explore in the
future.}

Note that for a large subset of the queries, attribute information is
not used, and only the topology is important.  Thus, we can store the
vertex attributes in a separate collection (column store), removing
the attribute map and replacing it with existence bit sets instead.
This is the essence of the special case of MultiGraph, called {\bf
  MultiGraphColumn (MGC)}, depicted in Figure~\ref{fig:mgc}.  The MGC
representation allows storage of an arbitrary number of vertex
attributes without using complex per-vertex lists, read from disk only
as needed, and so ammenable to lazy evaluation, an important
performance optimization in Apache Spark.  Further compression can be
achieved by storing vertex attributes values only once across all time
periods in which they are the same, similar to how temporal databases
represent this type of data (e.g., see~\cite{Muller2008}).  The
drawback of this approach is that decompression is required to
support, for example, the \insql{TGroup} operation.

\begin{figure}[t!]
\includegraphics[width=3.2in]{figs/mgc.pdf}
\caption{MGC representation of T1 from Figure~\ref{fig:tg}.}
\label{fig:mgc}
\end{figure}

{\bf OneGraph (OG).}  The most topologically compact representation is
to store each vertex {\em and} each edge only once for the whole
evolving graph, by taking a union of the snpashot vertex and edge
sets.  The OneGraph data structure, or OG for short, uses this
representation in our system.  Similar to MG, vertex and edge
attributes are stored in time-indexed maps (Figure~\ref{fig:og}).
Compared to MG, this leads to 75\% storage savings.  The OG data
structure provides some benefits in addition to compactness, since it
reduces the total communication between vertices in Pregel-based
analytics in batch mode.  The drawback is that OG is much denser than
individual snapshots.  As with MG, \insql{TSelect} is a subgraph
operation, and \insql{TGroup} is a transform and filter operation --
both for vertices and edges.

\begin{figure}[t!]
\includegraphics[width=3.2in]{figs/og.pdf}
\caption{OG of T1 from~\ref{fig:tg}.}
\label{fig:og}
\end{figure}

Similar to MGC, {\bf OneGraphColumn (OGC)} uses a single graph to
represent the union of vertices and edges, with bit sets for presence
information, while attribute information is stored separately.  This
is not as compact as storing attributes within the graph elements, but
is faster in many operations where only graph topology is required.

\subsection{Partitioning Strategies}  
\label{sec:sys:partition}

Graph partitioning can have a tremendous impact on system performance.
A good partitioning strategy needs to (1) be balanced, assigning an
approximately equal number of units (in our case, edges) to each
partition, and (2) limit the number of cuts (in our case, vertex cuts)
across partitions, to reduce cross-partition communication.

We support six different edge partitioning strategies, which are
applied prior to the operation but after loading and can be re-applied
at any point.

{\bf Canonical Random Vertex Cut (CRVC).}  The source and destination
ids of a vertex are hashed in a canonical direction, as a tuple, and
the result is distributed among the available partitions.  The result
is a random vertex cut that colocates all edges that connect a given
pair of vertices, regardless of direction.  This strategy is available
in GraphX and was used without modification.

{\bf 2D Edge Partitioning (E2D).}  A sparse edge adjacency matrix is
partitioned in two dimensions (Figure~\ref{fig:e2d}), guaranteeing a
$2 \sqrt{n}$ bound on vertex replication, where $n$ is the number of
partitions. E2D can provide good performance for Pregel-style
analytics.  This strategy is also available in GraphX and was used
without modification.

\begin{figure}[t!]
\includegraphics[width=3.2in]{figs/E2D.pdf}
\caption{EdgePartition2D strategy over 4 partitions.}
\label{fig:e2d}
\end{figure}

{\bf Naive Temporal.}  Provided there are more time intervals in the
graph than there are partitions, each edge is placed in the time index
modulo number of partitions place, round-robin fashion.  In the case
where the graph covers a small time interval, multiple partitions are
used for each interval (we term this a {\em run}), and the CRVC
strategy is applied within each run.

{\bf Consecutive Temporal.}  If there are more time intervals than
partitions, consecutive time intervals are assigned to the same
partition in runs.  If there are more partitions, each time interval
is assigned an equal number of consecutive partitions, within which
CRVC strategy is applied (Figure~\ref{fig:consecutive}).  In many
networks earlier snapshots are smaller, sometimes significantly so,
than the later ones.  This partition strategy thus may result in an
unbalanced partitioning for networks with large skew.  The number of
vertex cuts depends on the rate of change between snapshots.  If
vertices persist across many time intervals, this strategy is not a
good choice.

\begin{figure}[t!]
\includegraphics[width=3.2in]{figs/Consecutive.pdf}
\caption{Consecutive strategy over 4 partitions and 3 time intervals.}
\label{fig:consecutive}
\end{figure}

{\bf Hybrid Random Cut Temporal.}  As the name implies, hybrid
strategies combine elements of temporal and structural criteria.  In
this strategy, time intervals are broken into runs, the width of which
depends on the operation.  For example, for the \insql{TGroup}
operation, the width of the run is the number of snapshots that are
grouped.  Equal number of partitions is assigned to each run, and
within the run edges are assigned using the CRVC strategy.  This
strategy was specifically designed for time-based operations such as
aggregation to co-locate edges that are to be grouped.

{\bf Hybrid 2D Edge Temporal.}  Similar to the hybrid above, this one
also uses runs, but within the run uses the 2D Edge partitioning
strategy instead. Figure~\ref{fig:hybrid2d} shows an example of this
strategy with 4 time intervals in runs of 2.

\begin{figure}[t!]
\includegraphics[width=3.2in]{figs/Hybrid2D.pdf}
\caption{Hybrid 2D Edge Temporal strategy over 4 partitions and 4 time
  intervals in runs of width 2.}
\label{fig:hybrid2d}
\end{figure}

Not all partition strategies can be applied to every data structure.
For example, only purely structural strategies can be applied to OG.
We report on the experimental effectiveness of the strategies in
Section~\ref{sec:exp}.

\subsection{Query Optimization}
\label{sec:sys:optimization}

The declarative nature of \ql means that, like in SQL, queries can be
optimized for better performance.  \ql optimizations can be done by
operator reordering, application of best data structure and partition
strategy based on rules, and cost-based optimizations.  We now discuss
the first two options.

{\bf Operator reording.} Recollect that the logical order of
evaluation of a \ql query is selection, join, aggregation, and,
finally, analytics and projection.  We can show that there are classes
of queries which can be reordered without the loss of correctness.

{\bf Aggregation before join.}  Temporal aggregation produces a
smaller number of snapshots than its input.  Since the performance of
joins depends on the number of snapshots being structurally joined,
smaller number of snapshots will lead to faster performance.  Consider
this query:

\begin{small}
\begin{verbatim}
    TSelect Any V; Any E
    From    T1 TOr T2
    TGroup  by 5 years
\end{verbatim}
\end{small}

Assuming that T1 and T2 have 1-year resolution, \insql{TGroup} will
produce 20\% fewer graphs to join.  This reordering cannot be applied
every time, however, because an invalid result can be produced.  For
example, if T1 and T2 are those in Figures~\ref{fig:tg}
and~\ref{fig:tg_t2}, then an aggregation would produce graphs with
temporal schema ([2010, 2015), [2015, 2020)) for T1 and ([2008, 2013),
      [2013, 2018)) for T2, which are not union-compatible.  We check
        that the temporal schema of aggregated graphs is
        union-compatible and apply the reordering only if it is.

{\bf \insql{TAnd} join before temporal selection.}  Temporal
intersection between two graphs with a small temporal overlap can cut
down on graph loading time significantly.  Consider this query:

\begin{small}
\begin{verbatim}
   TSelect All V; All E
   From    T1 TAnd T2
\end{verbatim}
\end{small}

There is an implicit temporal selection here over the entire T1 and T2
duration.  If the T1 and T2 temporal bounds are known, then the query
can be rewritten with an explicit temporal selection matching the
length of the overlap.  In the case of example graphs T1 and T2 from
Section~\ref{sec:example}, we can rewrite the query like this:

\begin{small}
\begin{verbatim}
   TSelct All V; All E
   From   T1 TAnd T2
   TWhere T1.Start >= 2010 And T2.End <= 2014 
          And T2.Start >= 2010 And T2.End <= 2014
\end{verbatim}
\end{small}

This rewriting reduces the loading size of T1 and T2 by 30\% and,
since loading time is linearly dependent on loading size (as we show
in the next section), reduces the overall query time by about the same
amount.

{\bf Projection before aggregation or join.}  Recollect that
aggregation and joins operate not only on the graph structure, but
also graph attributes.  In some cases, the values of some or all of
those attributes are irrelevant to the query, and the attribute
aggregation can be greatly reduced or eliminated.  Consider query Q6
from Section~\ref{sec:example}, reproduced here for convenience:

\begin{small}
\begin{verbatim}
     TSelect   All V[vid, pagerank() as pr]; 
               Any E[vid1, vid2, sum(cnt)]
     From      T1 TOr T2
\end{verbatim}
\end{small}

T1 and T2 vertices have two attributes: name and salary, but they are
irrelevant to this particular query.  We can project T1 and T2 to
\insql{V [vid]} prior to evaluation of the \insql{TOr} operation for
better performance.  However, we cannot apply analytics in projection
before the other operations, so we cannot always perform projection
first.

{\bf Data Structure selection.} In addition to selecting the most
efficient order of operations, we select the best data structure for
the query.  Our analysis of data structures' effectiveness for each
individual operation is shown in the next section.  Switching between
data structures from operation to operation is possible but
computationally inefficient.  In general, however, SnapshotGraph
performs better on graphs with a small temporal span, while OneGraph
on those with large span.

\eat{Instead, we pick the structure that
provides the best compromise for the query operations by associating
costs with each operation and picking the least expensive one.}

{\bf Partitioning.} In our experience, the partitioning of the data
affects the performance more than the data structure choice itself.  A
poor partition strategy, in terms of the correct number of partitions
and the allocation of edges to partitions, can be 10x times slower
than a good one, as we show in the next section.  We evaluated
different strategies and partition numbers experimentally and this
serves as a basis of simple rules for partition selection.  For
example, EdgePartition2D strategy provides a significantly better
performance for snapshot-based structural analytics such as pagerank.

Other criteria that can affect the query performance are data skew,
change rate between snapshots, and data size.  We continue to evaluate
these aspects and will incorporate them into the query optimization.

