\section{Model}
\label{sec:model}

In this section we first formally define snapshot graphs, and then
show how graph evolution is represented by assigning temporal meaning
to sequences of snapshot graphs.

\begin{definition}[Snapshot graph]
\label{def:sg} 
A {\em snapshot graph} (or a {\em snapshot}) is a pair $G = (V,E)$,
where $V$ is a finite set of nodes with schema $(\underline{vid},
a_1, \ldots, a_n)$, and $E$ is a finite set of edges connecting
pairs of nodes from $V$, with schema $(\underline{vid_1},
\underline{vid_2}, a_1, \ldots, a_m)$.
\end{definition}

\begin{figure}
\includegraphics[width=3.2in]{figs/snapshot.pdf}
\caption{Snapshot of VLDB co-authorship graph for 1940, with a single string node attribute $name$ and no edge attributes.}
\label{fig:sg}
\end{figure}

Attributes of vertices and of edges are not restricted to be of atomic
types, but may, e.g., be maps or tuples. However it is required that
all vertices (resp. edges) of $G$ have the same schema, i.e., $V$ and
$E$ are homogeneous sets.

\begin{definition} [Structural union-compatibility]
\label{def:scompat}
Snapshot graphs $G' = (V', E')$ and $G'' = (V'', E'')$ are
union-compatible if $V'$ and $V''$ are union-compatible, and $E'$ and
$E''$ are union-compatible.
\end{definition}

This is the standard union compatibility definition, which requires
that vertex schema $V$ and edge schema $E$ be the same for $G'$ and
$G''$.  For example, the snapshot in Figure~\ref{fig:sg} is only
compatible with other snapshots where vertices have one non-key string
attribute $name$ and no non-key edge attributes.

$G$ may represent a directed or an undirected graph.  For undirected
graphs we choose a canonical representation of an edge, with $vid_1
\leq vid_2$ (self-loops are allowed).

We next describe how time is represented in our model.  Following the
SQL:2011 standard~\cite{DBLP:journals/sigmod/KulkarniM12}, we adopt
the {\em closed-open} period model, where a period represents all
times starting from and including the start time, continuing to but
excluding the end time.

\begin{definition}[Time period]
\label{def:period} 
A {\em time period} \\$p = [start, end)$ is an interval on the timeline,
  subject to the constraint $start < end$.  We refer to the length of
  time covered by $p$ as its {\em resolution}.
\end{definition}

We focus on {\em valid time}, represented by {\em application-time
  period} in SQL:2011 --- the time period during which data is
regarded as correctly reflecting reality.  This is in contrast to {\em
  transaction time} (or {\em system-time period}), which refers to the
time period during which a row is committed to the database.  Our goal
in this work is to support complex analytics over evolving graphs,
under the assumption that all historical data is available in the
database and is read-only.

We represent graph evolution by associating a sequence of snapshots,
which are not themselves time-aware, with a a sequence of time
periods.  This is stated formally next.

\begin{definition} [Temporal sequence]
\label{def:tseq} 
A {\em temporal sequence} $P = (p_1, \ldots, p_n)$ is a
sequence of consecutive non-overlapping time periods of the same
resolution, with no gaps.  That is,

\begin{enumerate}
\item $\forall i < n, p_i.end = p_{i+1}.start$, and 
\item $\forall i, j, p_i.end - p_i.start = p_j.end - p_j.start$.
\end{enumerate}
\end{definition}

$P$ may be equivalently described by 3 values: the start of the
earliest period $P.start = p_1.start$, the end of the latest period
$P.end = p_n.end$, and the resolution of any period $P.res = p_1.end -
p_1.start$. For convenience, we refer to the number of periods in the
sequence as $P.size$.  

For example, $P=([1940,1945), \ldots, [2010,2015))$ represents a
    temporal sequence with $P.start=1945$, $P.end=2015$, $P.res=5$,
    and $P.size=15$.

A special sequence $P^{\epsilon}$ is the null sequence, with
$P^{\epsilon}.res=null$, $P^{\epsilon}.start=null$,
$P^{\epsilon}.end=null$, and  $P^{\epsilon}.size=0$.

\eat{\vera{According to the wiki, $[a,a)$ is considered an empty
      set. So if we just follow the standard interval math semantics,
      we can say: A null temporal sequence is a sequence represented
      by the $[p.start,p.end)$ time interval regardless of the
        resolution. By definition it is of size 0.}}

We next define union-compatibility for temporal sequences, and present
two basic operations on temporal sequences, which we will use as
building blocks when we define our query language in
Section~\ref{sec:lang}.

\begin{definition} [Temporal Union-Compatibility]
\label{def:tcompat} 
Temporal sequences $P'$ and $P''$ are union-compatible if they have
the same resolution, and if we can construct a valid temporal sequence
$P$ with $P.start = min(P'.start, P''.start)$, $P.end = max(P'.start,
P''.start)$, and $P.res = P'.res$.  $P^{\epsilon}$ is union-compatible
with any temporal sequence.
\end{definition}

For example, consider sequences $P_1=([1,3),[3,5),[5,7))$,
      $P_2=([9,10),[10,11))$, $P_3=([8,10),[10,12),[12,14))$,
                $P_4=([12,14),[14,16))$, and $P_5=([20,22))$.
$P_2$ is not union-compatible with any other sequence because it has a
different resolution.  $P_1$ and $P_3$ are not union-compatible
because, while $P_1.res = P_2.res = 3$, it is not possible to
construct a valid temporal sequence with $P.start=1$, $P.end=14$ and
$P.res=3$.  Finally, $P_1$, $P_4$ and $P_5$ are pair-wise
union-compatible.  As our example illustrates, a pair of union-compatible sequences may
or may not overlap, and may not even be consecutive.

\begin{definition} [Temporal Intersection]
\label{def:tseqand}
Temporal intersection of union-compatible sequences $P'$ and $P''$,
denoted $P' \cap P''$, is a time sequence $P$, containing intervals
that are in common to $P'$ and $P''$.  If no intervals are in common
to $P'$ and $P''$, this operation returns $P^{\epsilon}$.
\end{definition}

For example, $P_3 \cap P_4 = ([12,14))$ and $P_3 \cap P_5 =
  P^{\epsilon}$.

\begin{definition} [Temporal Union]
\label{def:tseqor}
Temporal union of union-compatible sequences $P'$ and $P''$, denoted
$P' \cup P''$, is the sequence $P$ with $P.start = min(P'.start,
P''.start)$, $P.end = max(P'.start, P''.start)$, and $P.res = P'.res$.
\end{definition}

For example, $P_3 \cup P_4 = ([8,10),\ldots,[14,16))$ and
$P_3 \cup P_5 = ([8,10), \ldots, [20,22))$.

Recall that a snapshot represents a single state of an evolving graph,
and is not itself time-aware.  Temporal evolution of a graph is
represented by a sequence of snapshots, called {\em temporal graphs}
in our formalism.

\begin{definition} [Temporal Graph]
\label{def:tgraph} 
A {\em temporal graph} (or a {\em \tg}) $T = (G_1, \ldots, G_n; P)$
  associates a sequence of $n$ structurally union-compatible snapshots
  with a temporal sequence $P$, such that $P.size = n$.
\end{definition}

Snapshot graphs in the sequence define the {\em structural schema} of
$T$, while $P$ specifies the {\em temporal schema} of $T$.

Importantly, identity of a vertex persists across snapshots in a
\tg, and across \tgs.  For example, in Figure~\ref{fig:tgraph}
vertex with id 3 represents the same author W. V. Quine, who published
both in 1940 and in 1952.

\begin{figure}
\label{fig:tgraph}
\includegraphics[width=3.2in]{figs/temporalgraph.pdf}
\caption{A \tg of VLDB co-authorship over the period from 1940 to 1953.}
\end{figure}

To conclude this section, we define union-compatibility for \tgs.

\begin{definition} [\tg Union-Compatibility]
\label{def:tuc} \tgs $T'$ and $T''$ are union-compatible if they are both
structurally union-compatible (per Definition~\ref{def:scompat}) and
temporally union-compatible (per Definition~\ref{def:tcompat}).
\end{definition}

The \tg of Definition~\ref{def:tgraph} is the basic element in our
model.  In what follows, we assume that a relation in our database
corresponds to a single temporal graph, not to a collection of
temporal graphs.  In the next section we will define the temporal
graph query language \ql, in which operators take as input a single
\tg or a pair of \tgs, and produce a \tg as output.




