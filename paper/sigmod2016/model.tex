\section{Model}
\label{sec:model}

In this section we first formally define snapshot graphs, and then
show how graph evolution is represented by assigning temporal meaning
to sequences of snapshot graphs.

\begin{definition}
\label{def:sg}
A {\em snapshot graph} (or a {\em snapshot}) is a tuple $G = (V,E)$,
where $V$ is a finite set of nodes with schema
$sch(V)=(\underline{vid}, attr_1, \ldots, attr_n)$, and $E$ is a
finite set of edges connecting pairs of nodes from $V$, with schema
$sch(E)=(\underline{vid_1}, \underline{vid_2}, attr_1, \ldots,
attr_m)$.
\end{definition}

Attributes $attr_i$ in $sch(V)$ and $sch(E)$ are not restricted to be
of atomic types, but may be, e.g., maps or tuples. However it is
required that all vertices (resp. edges) of $G$ have the same schema,
i.e., $V$ and $E$ are homogeneous sets.  

A pair of snapshots $G_1 = (V_1, E_1)$ and $G_2 = (V_2, E_2)$ are
union-compatible if and $V_2$ are union-compatible, and $E_1$ and
$E_2$ are union-compatible.

$G$ may represent a directed or an undirected graph.  For undirected
graphs we choose a cannonical representation of an edge, with $vid_1
\leq vid_2$ (self-loops are allowed).

A snapshot represents a single state of a an evolving graph, and is
not itself time-aware.  We next describe how temporal evolution of a
graph is represented by a sequence of snapshots, denoted {\em temporal
  graphs} in our formalism.  But first we must precisely state how
time is represented in our model.

Our representation of time follows the SQL:2011
standard~\cite{DBLP:journals/sigmod/KulkarniM12}.  We adopt the {\em
  closed-open} period model, i.e., a period represents all times
starting from and including the start time, continuing to but
excluding the end time.

\begin{definition}
\label{def:period}
A {\em time period} $p = [start, end)$ is an interval on the timeline,
  demarcated by a start time and an end time, subject to the
  constraint $start < end$.  We refer to the length of time covered by
  $p$ as its {\em resolution}.
\end{definition}

We focus on {\em valid time}, represented by {\em application-time
  period} in SQL:2011 --- the time period during which data is
regarded as correctly reflecting reality.  This is in contrast to {\em
  transaction time} (or {\em system-time period}), which refers to the
time period during which a row is committed to or recorded in the
database.  Our goal in this work is to support complex analytics over
evolving graphs, under the assumption that all historical data is
avaiable in the database and is read-only.

We represent graph evolution by associating a sequence of snapshots
with a {\em temporal schema} --- a sequence of consecutive
non-overlapping time periods, with no gaps.  Further, we require that
all time periods in the sequence have the same resolution.

\begin{definition}
\label{def:tseq}
A {\em temporal sequence} $P = (p_1, \ldots, p_n)$ is a
sequence of consecutive non-overlapping time periods of the same
resolution, with no gaps.  That is,

\begin{enumerate}
\item $\forall i \leq n, p_i.end = p_{i+1}.start$
\item $\forall i, j, p_i.end - p_i.start = p_j.end - p_j.start$  
\end{enumerate}

$P$ is fully described with three values: the start of the earliest
period $P.start = p_1.start$, the end of the latest period $P.end =
p_n.end$, and the resolution of any period $P.res = p_1.end -
p_1.start$. For convenience, we sometimes write $P(start, end, res)$.
\end{definition}

\begin{definition}
\label{def:tunion}
Temporal sequences $P'$ and $P''$ are union-compatible if:

\begin{enumerate}
\item $P'.res = P''.res$ and
\item Construct $P(min(P'.start, P''.start), max(P'.end, P''.end),$
  $P'.res)$.  Check that $P' \subseteq P$ and $P' \subseteq P$.
\end{enumerate}
\end{definition}

\begin{definition}
\label{def:tgraph}
A {\em temporal graph} $T = (G_1, \ldots, G_n; P)$ associates a
sequence of $n$ union-compatible snapshots with a temporal sequence
$P$, such that $P.size = n$.
\end{definition}

Snapshot graphs in the sequence agree on their schemas of vertices and
edges, and define a {\em structural schema} of $T$, while temporal
schema of $T$ is specified by $P$.  Identity of a vertex persists
across snapshots in a temporal graph, and across temporal graphs.

A {\em temporal graph} of Definition~\ref{def:tgraph} is the basic
element in our model.  In what follows, we assume that a relation in
our database corresponds to a single temporal graph, not to a
collection of temporal graphs.  We next define a temporal graph query
language \ql, in which operators take as input a single temporal graph
or a pair of temporal graphs, and produce a temporal graph as output.

To support binary operations, we next define union compatibility of
temporal graphs.

\begin{definition}
\label{def:tuc}
Temporal graphs $T'$ and $T''$ are union-compatible if both their
structural and their temporal schemas are union compatible. 
\end{definition}




