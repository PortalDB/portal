\section{Model}
\label{sec:model}

% Something about our representation of time.  Ints, temporal
% datatype, unit of time.  See what they do in temporal SQL.

We work with two types of schema: snapshot schema and temporal schema.

\begin{definition}
\label{def:sg}
A {\em snapshot graph} (or a {\em snapshot}) is a tuple $G = (V,E)$,
where $V$ is a finite set of nodes with schema $sch(V)=(vid, attr_1,
\ldots, attr_n)$, with $vid$ as key of $V$, and $E$ is a finite set of
edges connecting pairs of nodes from $V$, with schema $sch(E)=(vid_1,
vid_2, attr_1, \ldots, attrs_m)$, and with $(vid1,vid2)$ as its key.
\end{definition}

Note that attributes $attr_i$ in $sch(E)$ and $sch(A)$ are not
restricted to be of atomic types, but may be, e.g., maps or tuples.
$G$ may represent a directed or an undirected graph.  For undirected
graphs we choose a cannonical representation of an edge, with $vid1
\leq vid2$ (self-loops are allowed).

Snapshot graphs $G_1 = (V_1, E_1)$ and $G_2 = (V_2, E_2)$ are
union-compatible if $V_1$ and $V_2$ are union-compatible, as are $E_1$
and $E_2$.

For example, ....

A snapshot represents a single state of a an evolving graph, and is
not itself time-aware.  We next describe how temporal evolution of a
graph is represented by a sequence of snapshots.

\begin{definition}
\label{def:tg}
A {\em temporal graph} $T = (S_{1..n}, f)$ associates a finite
sequence of union-compatible snapshot graphs $S_{1..n} = <G_1, \ldots,
G_n>$ with a function $f$, which maps an index of a graph in the
sequence to a time interval $[t_s, t_e)$.  Graphs in the sequence
  cover consecutive non-overlapping time intervals of equal size, that
  is:

\begin{enumerate}
\item $\forall i, f(i).t_e = f(i+1).t_s$
\item $\forall i, f(1).t_s - f(1).t_e = f(i).t_s - f(i).t_e$  
\end{enumerate}
\end{definition}

For convenience, we refer to $ f(i).t_s - f(i).t_e$ as the {\em
  resolution} of $T$, denoted $r_T$, or simply $r$ when $T$ is clear
from context.

Importantly, identity of a vertex, denoted by $vid$, is global, and is
preserved across snapshots in a temporal graph, and across temporal
graphs.

For example, ..... 
% Point out how a vertex persists, and an edge persists or doesn't.

A {\em temporal graph} is the basic element in our model.  In what
follows, we assume that a relation in our database corresponds to a
single temporal graph, not to a collection of temporal graphs.  We
next define a temporal graph query language \ql, in which operators
take as input a single temporal graph or a pair of temporal graphs,
and produce a temporal graph as output.  To support binary operations,
we next define union compatibility of temporal graphs.

\begin{definition}
\label{def:tuc}
Temporal graphs $T' = (S'_{1..n}, f')$ and $T' = (S''_{1..m}, f'')$ are
union-compatible if
\begin{enumerate}
\item snapshot graphs in sequences are compatible
\item resolutions are the same
\item intervals align: create T with [min(start), max(end)) at
  resolution r, does there exist a mapping from intervals of T' and T"
  to T?
\end{enumerate}
\end{definition}

For example: a pair that is not union-compatible because resolution is
different. (Because we don't know how to map edges / vertices within
an interval).  Example of a pair that is not union-compatible although
the sequences don't overlap.


