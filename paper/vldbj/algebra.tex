\section{Algebra}
\label{sec:algebra}
\setlength{\textfloatsep}{5pt}% Remove \textfloatsep

\begin{table}
\caption{\tga operators.}
\small
\begin{tabular}{l | c}
\hline
$\slice{\bc}{\ttt}$ & slice \\ \hline
$q^T_v(\tve)$ & vertex subgraph \\ \hline
$q^T_e(\tve)$ & edge subgraph \\ \hline
$\vmap{F_v, \tve}$ & vertex map \\ \hline
$\emap{F_e, \tve}$ & edge map \\ \hline
$\insql{agg}^T$ & aggregation \\ \hline
$\cup^T$ & union \\ \hline
$\cap^T$ & intersection \\ \hline
$\setminus^T$ & difference \\ \hline
$\insql{node}^T_a$ & attribute node creation \\ \hline
$\insql{node}^T_w$ & window node creation \\ \hline
$\insql{edge}^T$ & edge creation \\ \hline
\end{tabular}
\label{tab:operators}
\end{table}

\tg algebra, or \tga for short, is compositional: operators take a \tg
or a pair of \tgs as input, and output a \tg.  Our goal in specifying
this algebra is completeness -- we provide temporal versions of common
compositional graph operators and add operators that enable
evolutionary analysis.  We specify the semantics of \tga using tuple
relational calculus and by showing a translation of each operator into
a sequence of Temporal Relational Algebra (\tra) expressions (with
nesting, to accommodate non-1NF vertex/edge attributes).  The mapping
to \tra gives us the ability to support \tga in any RDBMS.  However,
the \tga operators can also be supported by other access methods, as
we show in Section~\ref{}.

\subsection{Preliminaries}
\label{sec:algebra:tra}

\tra extends relational algebra by adding a temporal attribute,
usually with tuple stamping, and specifying how relational operators
are applied to temporal relations.  Operators are considered temporal
if they generate temporal relations when applied to temporal
relations.  \tra may have point or sequenced semantics which dictates
properties of the operators~\cite{Bohlen1998,Dignos2012}.  Both point
and sequenced semantics require snapshot reducibility and extended
snapshot reducibility properties, and sequenced semantics additionally
requires change preservation.

Snapshot reducibility means that a temporal operator applied to a
temporal relation produces the same result as an equivalent
non-temporal operator over corresponding snapshots.  Extended snapshot
reducibility allows explicit references to timestamps in the operators
by propagating them as data.  Change preservation dictates how time
points in the result are combined into periods.  With point semantics,
value-equivalent consecutive and overlapping tuples are merged into a
single tuple.  Change preservation, however, states that operators
only merge contiguous time points of a result if they have the same
lineage.  See~\cite{Dignos2012} for formal definitions.  We explore
some of the implications of applying sequenced semantics to evolving
graphs in~\cite{MoffittEDBT2017}, but in this work we adhere to point
semantics.

\eat{In Section~\ref{sec:algebra:integrity} we present the primitives
  that are needed to enforce soundness of \tga.  Then, in
  Sections~\ref{sec:algebra:unary} through~\ref{sec:algebra:ecreate},
  we present \tga operators.  Section~\ref{sec:analytics} presents an
  extension of \tga to support Pregel-style analytics.}

\subsection{Unary operators}
\label{sec:algebra:unary}

{\bf Slice.} The slice operator, denoted $\slice{\bc}{\ttt}$, where
$\bc$ is a time interval, cuts a temporal slice from \ttt. The
resulting \tg will contain vertices and edges whose period $\bp$ has a
non-empty intersection with $\bc$: $\{ (v, \bp \cap \bc)~~|~~(v, \bp) \in \tv
\wedge (\pred{\bc}{overlaps}{\bp} \vee \pred{\bc}{contains}{\bp})
\}$, and analogously for \te.

We translate this \tga operator to TRA statements over each
constituent relation of \tve: $\slice{\bc}{\tv}$ and similarly for
\te.

Slice is a common operator in temporal algebras, although its exact
semantics varies.  It does not have a nontemporal equivalent.  Note
that slice is only valid in point semantics because it modifies
individual time stamps and thus violates change preservation property.

%: $\slice{\bc}{\tv} = \{ (v, \bp \cap \bc)~~|~~(v, \bp) \in \tv
%\wedge (\pred{\bc}{overlaps}{\bp} \vee \pred{\bc}{contains}{\bp})
%\}$, and analogously for each \te, \tav and \tae.  apply select
%operator to each of the four constituent relations of \tve: $\forall
%x \in \{\tv,\te,\tav,\tae \}, x' = \pi_{A,p \cap c}(\sigma_{p \cap c}
%(x))$, where $A$ is the non-temporal schema of $x$.

\eat{If $p.start < c.start$ or $p.end > c.end$ for some tuple $(g,
  p)$, then $p$ is trimmed to be within the boundaries of $c$: $\tau_c
  (\trg) = \{ (g, p \cap c)~~|~~(g, p) \in \trg \wedge
  (\pred{c}{overlaps}{p} \vee \pred{c}{contains}{p})\}$.  }

%Slice does not violate any of the soundness requirements {\bf R1} ---
%{\bf R4}. \eat{, see Appendix~\ref{sec:app1} for proofs of this and
%  similar statements.}

\begin{example}
\label{ex:slice}
Using T1 \tg from Figure~\ref{fig:tg_ve}, let
$\slice{[2015/5,2015/8)}{\ttt1}$ compute a slice from May to August
  of 2015.  The result is shown below.
\begin{table}[h]
\begin{tabular}{c | c | c}
\hline
{\bfseries{\underline v}} & {\bfseries{\underline p}} & {\bfseries a} \\ \hline
v1 & [2015/5,2015/7) & name=Alice,school=Drexel \\ \hline
v2 & [2015/5,2015/8) & name=Bob,school=CMU \\ \hline
v3 & [2015/5,2015/8) & name=Cathy,school=Drexel \\ \hline
\end{tabular}
\end{table}
\end{example}

{\bf Subgraph.} Temporal subgraph matching is a generalization of
subgraph matching for non-temporal graphs~\cite{Wood2012}.  This query
comes in two variants.

\eat{A temporal vertex-subgraph query $q^T_v(\tve)$ is a conjunctive query
that takes $\tve (\tv, \te, \tav, \tae)$ as input, and outputs
$\tve'$, an induced temporal subgraph of \tve in which vertices are
defined by $q^T_v$.}

Temporal vertex-subgraph \subv{q^t_v}{\ttt} computes an induced
subgraph of \tve $\tve'(\tv', \te')$, with vertices defined by the
temporal datalog query (TDQ) $q^t_v$:  
Note that this is a
subgraph query, and so $\tv' \subseteq^T \tv$.

Temporal edge-subgraph \sube{q^t_e}{\ttt} computes a subgraph of \ttt
$\tve'(\tv', \te')$ in which edges are defined by TDQ $q^t_e$.  Since
this is a subgraph query, $\te' \subseteq^T \te$.

Queries $q^t_v$ and $q^t_e$ may use either or both of the constituent
relations of \tve, and may explicitly reference temporal information,
and so require both input relations to be
coalesced~\cite{DBLP:reference/db/Bohlen09}.

Many graph languages include subgraph operations, most commonly on the
attributes of the individual entities (conjunctive
query)~\cite{Wood2012}.  We generalize this to using datalog queries
with the generality goal in mind such that regular path queries and
conjunctive regular path queries are supported.  This allows the user
to, for example, select only vertices with non-zero degree as a single
operation or only vertices connected by particular paths, including
using temporal predicates.

\eat{Following the computation of $\tv' = q^t_v(\tv)$, \subv{q^t_v}{\ttt}
must invoke $\coal{\tv'}$ to enforce {\bf R1} and {\bf R3}; and
$\constr{\te'}{\tv'}$ to enforce {\bf R2}.}
%
\eat{Following the computation $\te' = q^t_e(\te)$, \sube{q^t_e}{\ttt} must
invoke $\coal{\te'}$ to enforce {\bf R1} and {\bf R3}.}

\eat{
$\sigma_{C_V,C_E} (\ttt) = (\tv',\te',\tav',\tae') ~|~$ \\
$\tv' = \pi_{v,p} (\sigma_{C_V} ( \tv \bowtie^T_v \tav))$ \\
$\te' = \pi_{v_1,v_2,p} (\sigma_{C_E} ( \tae \bowtie^T_{v_1} \tav \bowtie^T_{v_2} \tav))$ \\
$\tav' = \sigma_{C_V} (\tav)$ \\
$\tae' = \pi_{v_1,v_2,a} (\sigma_{C_E} (\tae \bowtie^T_{v_1} \tav \bowtie^T_{v_2} \tav))$}

\eat{ Note that as B{\"{o}}hlen showed
  in~\cite{DBLP:reference/db/Bohlen09}, correctness of a select
  operator with a temporal predicate depends on the coalesced state of
  the relation.  When composing a triplet, we carry the vertex and
  edge timestamps in their entirety as data to produce correct
  results.}

% There is no need for an example here, it is clear what these queries
% compute.

\eat{Because we allow predicates over the triplets, interesting conditions
can be expressed.  For example, we can filter out any edges where the
value of some property of its two vertices does not match.  Assuming
vertices have property \insql{group}, we can compute
$\sigma_{T,v1.a.group=v2.a.group}(\ttt)$.}

%\subsection{Temporal map}
%\label{sec:algebra:project}

{\bf Map.}  Temporal vertex-map and edge-map apply user-defined map
functions $f_v$ and $f_e$ to vertex or edge attributes.  Temporal
vertex-map $\vmap{f_v, \tve}$ outputs $\tve'$ in which
$\tv'=\pi^T_{v,f_v(a)}\tv$ and $\te'=\te$. Temporal edge-map
$\emap{f_e, \tve}$ is defined analogously.

\eat{Temporal map iterates over the tuples of \trg, and applies the
user-specified map functions $M_V$ and $M_E$ to the vertices and edges
of each $g$: $\map_{M_V,M_E} (\trg) = \{(g', p)~~|~~(g,p) \in \trg
\wedge g'= \map_{M_V,M_E}(g)\}$.
}  

While $f_v$ and $f_e$ are arbitrary user-specified functions, there
are some common cases.  Map may specify the set of properties to
project out or retain, it may aggregate (e.g., \insql{COUNT}) or
deduplicate values of a collection property, or flatten a nested
value.
%
To produce a valid \tg, $\vmap{f_v, \tve}$ must invoke $\coal{\tv'}$
and $\emap{f_e, \tve}$ must invoke $\coal{\te'}$.

% because this is an arbitrary operation, we don't need to invent
% syntax here.  it's also clear what this operation does, I don't
% think there is a need for an example.

\eat{In such cases we will use short-hand
notation similar to projection, listing the properties that we wish to
retain. For example, $\map_{M_V:{school},M_E:\emptyset} (\insql{T1})$
will keep only the school property of the vertices, and no properties
of the edges.  Another useful map operation eliminates duplicates in
the bag of a particular vertex or edge property.  \eat{It may also be
useful to flatten nested bags or aggregate multiple values of the same
property of a vertex or edge, e.g., compute a sum or an average
following temporal intersection or union
(Section~\ref{sec:algebra:join}).}}

\subsection{Aggregation}
\label{sec:algebra:agg}

Aggregation is a common graph operation that computes the value of a
vertex property $pname$ based on information available at the vertex
itself, at the edges associated with the vertex, and at its immediate
neighbors.  Aggregation can be used to compute simple properties such
as in-degree of a vertex, or more complex ones such as the set of
countries in which the friends of $v$ live.

It is convenient to think of aggregation as operating over a temporal
view $L(v_1,v_2,e,v_1.a,v_2.a,e.a,p)$, where $v_1$ refers to the
vertex for which the new property is being computed, $v_2$ refers to
the vertex from which information is gathered, $e$ refers to the edge
connecting $v_1$ and $v_2$, $v_1.a$, $v_2.a$ and $e.a$ are attributes
of the vertices and of the edge, and $p$ is the associated time
period.  $L$ is computed with a temporal join of \te with two copies
of \tv, one for each side of the edge

When \tve represents a directed graph, direction of the edge can be
accounted for in the way the join is set up (e.g., mapping $v_2$ in
\te to $v_1$ in $L$ if the goal is to aggregate information on
incoming edges).  When \tve represents an undirected graph (recall
that we choose a canonical representation of an edge, with $v_1 \leq
v_2$), or when direction of the edge is unimportant, $L$ can be
computed from $\te(v_1,v_2,p) \cup^T \te(v_2,v_1,p)$ rather than from
$\te$.

Aggregation is denoted $\insql{agg}^T(dir,cond,f_m,f_a,pname,\ttt)$,
where $dir$ is the direction of the edge (one of 'right', 'left' or
'both') that determines how $L$ is computed, $cond$ is a predicate
over $L$, $f_m$ is a map function that emits a value for each tuple in
the result of $\sigma^T_{cond}(L)$ (e.g., 1 for computing degree of
$v_1$, or $v_2.a.country$ for computing the set of countries in which
the friends of $v_1$ live).  Finally, $f_a$ is the function that
aggregates values computed by $f_m$, and $pname$ is the name of the
property to which the computed value is assigned.  Putting everything
together, and omitting the computation of $L$ for clarity: we compute
a temporal relation $R = \coal{v_1 \gamma^T_{f_a} (\pi^T_{v_1,f_m}
  (\sigma^T_{cond} L))}$. (Here, $\gamma^T$ is the temporal version of
relational aggregation, and $v_1$ is the grouping attribute.)  We then
compute an outer join of \tv with $R$, and invoke the resolve
primitive to reconcile the newly-computed property stored in $R.a$
with $\tv.a$: $\tv' = \resolve{\insql{set}(pname)}{\tv
  \rightouterjoin^T_{v=v_1} R}$. Note the use of the resolve primitive
at the last step.  Although there are no duplicates in the result of
the outer join of \tv and $R$, since $R$ is temporally coalesced and
the join is by key, resolve is needed to compute a bag-union of
properties in $R.a$ and $\tv.a$, and to aggregate the values
corresponding to $pname$ (in case $pname$ already occurred as a
property in $\tv.a$).

\eat{The result is a new isomorphic graph $\ttt'$:
$\agg{cond,msg,red}{\ttt} = \{ \tv, \te, \tav', \tae \}$, where $\tav'
= \pi^T_{v,a_1+a_2}(\tav \bowtie^T_v$ $_v\vartheta^T_{f} (msg
(\sigma_{cond} (\tae \bowtie^T_{v_1} \tav \bowtie^T_{v_2} \tav))))$.
The temporal join is used to add the new properties to the output
since it may have different periods of validity.  For example, while
each vertex in \tg may remain unchanged for the whole duration,
aggregating vertex degrees would result in an attribute value for each
period of topology change.}

We support various aggregation functions $f_a$, including the standard
\{ \insql{count} | \insql{min} | \insql{max} | \insql{sum} \}, which
have their customary meaning.  We also support \{ \insql{any} |
\insql{first} | \insql{last} | \insql{set} | \insql{list} \}, which
are possible to compute because properties being reduced have temporal
information.  \insql{first} and \insql{last} refer to the value of a
property with the earliest/latest timestamp, while \insql{set} and
\insql{list} associate a key with a collection of values.

\eat{As an example, to compute vertex in-degrees, we can use
  $\agg{msg=(dst,p,1),red=count}{\ttt}$.  To compute a set of places
  that all close friends have visited in the past year, assuming there
  is a property \insql{places} on friend vertices and closeness of
  friendship property on edges:\\ $\agg{cond=dst.p \cap [2015,2016) \&
      a.close > 0.8,msg=(src,p,dst.places)}{\ttt}$.}

%To produce a valid \tg, \insql{agg} must invoke \coal{\tav'}.
%\julia{I don't think so: {\bf R2}: require reduce function.}

\subsection{Binary set operators}
\label{sec:algebra:binary}

\begin{figure*}[t]
\begin{minipage}[b]{0.3\textwidth} %{2.0in}
\includegraphics[width=2.0in]{figs/T2_rel.pdf}
\caption{T2.}
\vspace{-0.5cm}
\label{fig:tg_t2}
\end{minipage}
\begin{minipage}[b]{0.3\textwidth} %{2.5in}
\includegraphics[width=2.5in]{figs/T1_union_T2_rel.pdf}
\vspace{-0.5cm}
\caption{$T1 \cup^T T2.$}
\label{fig:tg_union}
\vspace{-0.5cm}
\end{minipage}\hfill
\begin{minipage}[b]{0.3\textwidth} %{2.0in}
\includegraphics[width=2.0in]{figs/T1_inter_T2_rel.pdf}
\caption{$T1 \cap^T T2.$}
\label{fig:tg_inter}
\vspace{-0.5cm}
\end{minipage}
%\caption{Binary operators.}
%\label{fig:binary}
\end{figure*}

We support temporal versions of the three binary set operators
intersection ($\cap^T$), union ($\cup^T$), and difference
($\setminus^T$).

\eat{These \tra operators are not schema robust~\cite{Dignos2012} ---
  their result is affected if the argument relation is extended by an
  additional attribute.  This presents a problem when executing the
  set operations over the \tav and \tae relations as there is no
  guarantee that a vertex or an edge with the same identity and at the
  same time instant has the same attribute set.  Thus all three
  operators require that the resolve primitive be invoked as part of
  the computation.}

To compute $\insql{T1} \cup^T \insql{T2}$, we compute $\tv' =
\resolve{f_v}{\tv_1 \fullouterjoin^T_{v} \tv_2}$ and $\te' =
\resolve{f_e}{\te_1 \fullouterjoin^T_{e} \te_2}$.

Consider \insql{T1} in Figure~\ref{fig:tg_ve} and \insql{T2} in
Figure~\ref{fig:tg_t2}.  Figure~\ref{fig:tg_union} illustrates
\insql{T1} $\cup^T$ \insql{T2}.  According to the definition of
$\cup^T$, periods are split to coincide for any group, and thus the
attribute values for e.g., $v_1$ have three distinct tuples.

To compute $\insql{T1} \cap^T \insql{T2}$, we compute $\tv' =
\resolve{f_v}{\tv_1 \Join^T_{v} \tv_2}$ and $\te' =
\resolve{f_e}{\te_1 \Join^T_{e} \te_2}$.

As an example, when applying \insql{T1} $\cap^T$ \insql{T2}, only the
vertices and edges present in both \tgs are produced, thus eliminating
$v_3$ and $v_4$.  Period $[2/15, 4/15)$ for $v_2$ is computed as a
  result of the join of $[2/15, 5/15)$ in \insql{T1} and [$2/15,
      4/15)$ in \insql{T2}.  Figure~\ref{fig:tg_inter} illustrates
      \insql{T1} $\cap^T$ \insql{T2}.

To compute $\insql{T1} \setminus^T \insql{T2}$, we set $\tv' =
\pi_{\tv_1.v,p,\tv_1.a} \allowbreak (\sigma_{\tv_2.a = NULL}(\tv_1
\leftouterjoin^T_{v} \tv_2))$ and $\te' =\\
\pi_{\te_1.e,\te_1.v_1, \te_1.v_2,p,\te_1.a} (\sigma_{\te_2.a =
  NULL}(\te_1 \leftouterjoin^T_{e} \te_2))$.

To continue the example above, the result of \insql{T1} $\setminus^T$
\insql{T2} includes vertex v1 before 2/15 and after 6/15, splitting
one v1 tuple in \tv of T1 into two temporally-disjoint tuples in the
result.  See Figure~\ref{fig:tg_diff} in
Appendix~\ref{sec:app:examples} for full result.

Note that both $\cap^T$ and $\cup^T$ require that resolve be invoked,
to reconcile the vertex/edge attributes associated with vertices/edges
in the temporal intersection of the inputs.

\eat{As elsewhere, the default reduce function is \insql{set}.  In addition
to the functions defined in Section~\ref{sec:algebra:agg} we also
support \{ \insql{left} | \insql{right} \}, which select the attribute
of the left, resp. right, operand.}

\subsection{Node creation}
\label{sec:algebra:ncreate}

\begin{figure}[b]
\includegraphics[width=2in]{figs/agg3.pdf}
\vspace{-0.2cm}
\caption{\insql{node}$^T_a(school, set(school), max(cnt).)$}
\vspace{-0.4cm}
\label{fig:tg_agg3}
\end{figure}

The node creation operator enables the user to analyze an evolving
graph at different levels of granularity.  This operator comes in two
variants --- based on vertex attributes or based on temporal window.

{\bf Attribute-based node creation} is
denoted\\ $\insql{node}^T_a(g_1,\ldots,g_i,f_{v1}(k_1),\ldots,f_{vn}(k_n),\tve)$,
where $g_1,\ldots,g_i$ are the grouping attributes, and each
$f_{vj}(k_j)$ specifies an aggregation function $f_{vj}$ to be applied
to a vertex property $k_j$.  This operation allows the user to
generate a \tg in which vertices correspond to disjoint groups of
vertices in the input that agree on the values of all grouping
attributes.  For example, $\insql{node}^T_a(school,\tve)$ will compute
a vertex for each value of $\tv.a.school$.  Vertices that do not
specify a value for one or several grouping attributes at a given
time, will not contribute to the result for the corresponding
snapshot.

To compute $\tve' =
\insql{node}^T_a(g_1,\ldots,g_i,f_{v1}(k_1),\ldots,f_{vn}(k_n),\tve)$,
we generate the new vertex relation $\tv'$ by generating an id for
each group with a Skolem function and reconcile attribute values
within each group with the resolve primitive: $\tv' =
\resolve{f_{v1}(k_1),\ldots,f_{vn}(k_n)}{\pi^T_{skolem(g_1,\ldots,g_i), a} \tv}$.

Vertices of the input are partitioned on their values of the grouping
attributes.  Because the model supports multigraphs, all edges from
the input graph are retained, with their source and destination vertex
ids updated.  To compute the new edges relation $\te'$, we generate a
temporal conjunctive query that computes $E(e,v_1,v_2,p,a,v_1',v_2')$,
where $v_1'$ and $v_2'$ are the identifiers of the vertices in \tv' to
which $v_1$ and $v_2$ are mapped.  Finally, we compute $\te' =
\coal{\pi^T_{e,v_1',v_2',p,a} E}$.

Figure~\ref{fig:tg_agg3} illustrates attribute-based node creation
over \insql{T1} in our running example, with \insql{set(school)}
aggregation function for vertices and and \insql{max(cnt)} for edges.
Vertices $v_1$ and $v_3$ create a single new vertex $v_{10}$,
representing Drexel.

{\bf Window-based node creation} is denoted\\
$\insql{node}^T_w(w,q_v,q_e,f_{v1}(k_1),\ldots,f_{vn}(k_n),f_{e1}(l_1),\ldots,f_{em}(l_m),\tve)$,
where $w$ is the window specification, $q_v$ and $q_e$ are vertex and
edge quantifiers, and each $f_{vj}(k_j)$ ($f_{ej}(l_j)$) specifies an
aggregation function $f_{vj}$ (resp. $f_{ej}$) to be applied to a
vertex property $k_j$ (resp. edge property $l_j$).  This operation
corresponds to moving window temporal aggregation, and is inspired by
the stream aggregation work of~\cite{Li2005} and by generalized
quantifiers of~\cite{Hsu1995}, both adopted to graphs.

\eat{We argued in the introduction that it is interesting and
  insightful to analyze an evolving graph at different levels of
  granularity.  For example, the user may want to aggregate multiple
  consecutive representative graphs into a single representative
  graph, coarsening the granularity, or to predefine temporal
  resolution and look at the graph at that scale, irrespective of
  whether this resolution happens to be finer or coarse than the
  natural evolution rate of the graph.  For this, we introduce a node
  creation operator which is similar to the {\em moving window
    temporal aggregation} in temporal relational algebra.  Our
  approach is inspired by stream aggregation work of~\cite{Li2005},
  adopted to graphs, and by generalized quantifiers
  of~\cite{Hsu1995}.}

\eat{{\em Grouping attributes} $G_V$ are vertex properties by which
vertices are grouped into new entities, similar to \insql{GROUP BY}
clause in SQL.  Since node creation requires new identifiers, the
combination of the grouping properties can be used in a mechanism
equivalent to a Skolem function.  The simplest, default grouping
attribute is the $vid$ of the vertex.}

Window specification $w$ is of the form $n~\{unit|\insql{changes}\}$,
where $n$ is an integer, and $unit$ is a time unit, e.g., $10~min$,
$3~years$, or any multiple of the usual time units.  When $w$ is in
the form $n~\insql{changes}$, it defines the window by the number of
changes that occurred in \tve (affecting any of its constituent
relations). Window boundaries are computed left-to-right, i.e., from
least to most recent.

\eat{Our window specification by change is similar to slide-by-row window
in stream aggregation~\cite{Li2005}.  Note that, because \tg algebra
is compositional, we do not support node creation with overlapping
windows, because it does not produce a valid \tg.  To see why this is
so, consider applying a sliding window of 3 months range with 1 month
slide to graph \insql{T} in Figure~\ref{fig:tg_ve}.  We would produce
the following tuples for $v_1$: $(v_1, [1/15, 4/15), a_1)$, $(v_1,
  [2/15, 5/15), a_2)$, $(v_1, [3/15, 6/15))$, and so on, which clearly
      violates the temporally coalesced requirement in
      definition~\ref{tg}.}

\eat{Similar to~\cite{Li2005} we support creation simultaneously by
  time and by non-temporal attributes (e.g., vertex properties).  If
  the window specification is one change, then the operation devolves
  into pure structural reduce or node creation, as classified by
  Wood~\cite{Wood2012}.  If the grouping attribute is the vertex
  $vid$, then the operation is purely temporal, with no structural
  aspect.}

Vertex and edge quantifiers $q_v$ and $q_e$ are of the form \{
\insql{all} | \insql{most} | \insql{at least} $n$ | \insql{exists} \},
where $n$ is a decimal representing the percentage of the time during
which a vertex or an edge existed, relative to the duration of the
window (\insql{exists} is the default).  Quantifiers are useful for
observing different kinds of temporal evolution, e.g., to observe only
strong connections over a volatile evolving graph, we may want to only
include vertices that span the entire window ($q_v=\insql{all}$), and
edges that span a large portion of the window ($q_e=\insql{most}$).
 
\eat{The optional reduce functions compute new values for vertex and
  edge properties representative of the whole window, e.g.,
  $any(name), last(school), sum(cnt)$.}
%
 
\eat{Key-value pairs for vertex and edge properties for which no
aggregation functions are specified, are collected into a bag
corresponding to the entity in the result.  These can be subsequently
transformed with $map^T$ (Section~\ref{sec:algebra:project}).}

\eat{ 
Temporal aggregation over \tve follows the outline of
Algorithm~\ref{alg:op}, but requires an additional step, and is
revisited in Algorithm~\ref{alg:agg_ve}.
%
}

For both kinds of window specification (by unit or by number of
changes), we must (1) compute a mapping from a tuple in a temporal
relation to one or multiple windows, and (2) aggregate over each
window.  The $s$ parameter for the split primitive is the smallest
start date across \tve.

To compute $\tv'$, we apply split to \tv, resolve the attributes with
the aggregation functions, select only those vertices that meet the
quantification, and finally coalesce: $\tv' = \coal{\sigma^T_{q_V}
  (\resolve{f_{v1}(k_1),\ldots,f_{vn}(k_n)}{\wsplit{s}{w}{\tv}})}$.
$\te'$ is similar but with an additional constrain primitive
application when node quantification is more strict.

\eat{We compute group periods based on window
specification: $P = \textsf{computePeriods}(W, \tv, \te, \tav, \tae)$.
We use temporal aggregation and selection to evaluate $Q_V$ on each
group in $\tv$: $\tv' = \sigma_{P,Q_V}( _{G_V}\vartheta^T (\tv
\leftouterjoin^T \tav))$ and similarly for $\te$: $\te' =
\sigma_{P,Q_E}( _{G_V}\vartheta^T (\te \bowtie^T_{v_1=v} \tav
\bowtie^T_{v_2=v} \tav))$.  The edge triplets, obtained through
three-way temporal join of \tae with \tav are required to aggregate
edges by $G_V$.  }

\eat{
Node creation over \trg is computed by first calculating time periods
from $W$ and \trg, and then reducing and combining the representative
graphs directly.
}

\begin{figure*}[t]
%\begin{subfigure}[b]{0.5\textwidth}
\begin{tabular}{@{}c@{}}
%\begin{minipage}[b]{0.4\textwidth}
\includegraphics[width=2.0in,height=100pt]{figs/agg1.pdf} \\[\abovecaptionskip]
%\caption{By time: $w=3~\textsf{months}$.}
\small (a) By time: $w=3~\textsf{months}$.
\label{fig:tg_agg1}
%\end{subfigure}
\end{tabular}
%\end{minipage}\hfill
%\begin{subfigure}[b]{0.5\textwidth}
\begin{tabular}{@{}c@{}}
%\begin{minipage}[b]{0.4\textwidth}
\includegraphics[width=2.0in,height=100pt]{figs/agg2.pdf} \\[\abovecaptionskip]
%\caption{By change: $w=3~\textsf{changes}$.}
\small (b) By change: $w=3~\textsf{changes}$.
\label{fig:tg_agg2}
%\end{subfigure}
\end{tabular}
%\end{minipage}
\caption[]{Node creation,
  $\insql{node}^T_w(\mathsf{q_v=always},\mathsf{q_e=exists,f_v=\{first(name),~first(school)\}},\ttt)$.}
\vspace{-0.2cm}
\label{fig:tg_agg}
\vspace{-0.2cm}
\end{figure*}

Figure~\ref{fig:tg_agg1} illustrates window-based node creation by
time ($w=3~\textsf{months}$), and Figure~\ref{fig:tg_agg2} --- by
change ($w=3~\textsf{changes}$).  Both are applied to \insql{T1} in
our running example with \insql{all} quantifier for vertices and
\insql{exists} for edges, and \insql{first} aggregation function for
vertex and edge properties.  $v_2$ is present in the result in
Figure~\ref{fig:tg_agg1} starting at $4/15$ because it did not exist
for the entirety of the first window, while in
Figure~\ref{fig:tg_agg2} it is produced starting $6/15$.

\eat{Node creation may uncoalesce and requires FK enforcement.}

\eat{{\bf R1, R2}: coalesce every relation in $\ttt'$; {\bf R3}:
  enforce FK on $\tav', \te', \tae'$; {\bf R4}: require reduce
  function.}

\eat{ Our aggregation quantifiers are inspired by generalized
  quantifiers of~\cite{Hsu1995} with n-place delimiters.  $Q(R)$ as a
  Boolean-valued function of a relation''~\cite{Hsu1995}.  A
  quantifier contains an n-place determiner, e.g., ``at least one
  vertex in each window for each group'' is a 2-place determiner
  quantifier.  \tg algebra supports determiners from the set
  $\{at\ least\ one, all, most, at\ least\ n\}$, where $n$ is an
  integer representing a ratio.  $all$ is a usual universal quantifier
  that in standard SQL can be achieved with the use of two \insql{NOT
    EXISTS}.}

\subsection{Edge creation}
\label{sec:algebra:ecreate}

Edge creation
$\insql{edge}^T(q,f_{v1}(k_1),\ldots,f_{vn}(k_n),f_{e`}(l_1),\ldots,f_{em}(l_m),\ttt_1,\ttt_2)$
is a binary operator that computes a \tg on the union of the vertices
of $\ttt_1$ and $\ttt_2$, with edges and edge attributes computed by a
datalog query over the constituent relations of $\ttt_1$ and $\ttt_2$.
$\tv' = \resolve{f_{v1}(k_1),\ldots,f_{vn}(k_n)}{\tv_1 \cup^T \tv_2}$,
$\te' = \resolve{f_{e1}(l_1),\ldots,f_{em}(l_m)}{q(\ttt_1, \ttt_2)}$.

\eat{$\ecr{\ttt_1}{q,red}{\ttt_2}$ be an edge creation operator, where $q$
is a conjunctive query over constituent relations of $\tve_1$ and
$\tve_2$ and $red$ is a reduce function. $q$ must return a valid
temporal relation $(v_1, v_2, a_1, a_2)$.  The reduce function is used
to compute the final $\tae'$ relation.  $\ecr{\ttt_1}{q,red}{\ttt_2} =
\{ \tv', \te', \tav', \tae' \}$, where $\tae' =
\sigma^T_{v_1,v_2,red(a_1,a_2)}(q(\ttt_1, \ttt_2))$ subject to FK
constraint from $\tv_1 \cup^T \tv_2$, $\te' =
\sigma^T_{v_1,v_2}(\tae')$, \tv' is a subset of $\tv_1 \cup^T \tv_2$
such that it contains only vertices with edges in $\te'$, and $\tav'$
is an empty relation.  Intuitively, edge creation returns a new \tg
from nodes of $\ttt_1$ and $\ttt_2$ with no attributes, connected by
edges determined by $q$.}

Edge creation has several important applications.  It can be used to
compute friend-of-friend edges (passing in the same \tg as both
arguments).  Since $q$ can be recursive and include predicates over
the timestamps, $\insql{edge}^T$ can compute journeys.  A journey is a
path in the evolving graph with non-decreasing time
edges~\cite{Casteigts2011,Ferreira2004}.  By adding a temporal
condition to $q$, we can obtain journeys similar to time-concurrent
paths.

In graph theory, a graph join of two undirected unlabeled disjoint
graphs is defined as the union of the two graphs and additional edges
connecting every vertex in graph one with each vertex in graph two.
We can obtain a graph join by computing $\te' = \tv_1 \times^T \tv_2$.

SocialScope~\cite{Amer-Yahia2009} defines (non-temporal) graph
composition: compose the edge of the two operands and return an
edge-induced subgraph.  TGA can express this operator by a combination
of edge creation and vertex subgraph.

\eat{In \ql temporal graph composition can be computed using node
  creation of \ttt with itself and a $q = $ a temporal theta-join of
  $\tae$ and $\tae_2$. }

%\julia{Example of edge transpose goes here.}

\eat{Note that edge creation produces new edges.  To add these edges to the
original graph, a subsequent union must be performed.}

\eat{Graph composition operator in our algebra is a temporal extension of
the composition operator in SocialScope~\cite{Amer-Yahia2009}.  It
produces a graph induced by edges that are composed from edges in the
two operands for any time point when they coexisted.  The value of the
new edge attributes is determined by the resolve function, similar to
the set based operators.}

\eat{
Let $\odot_{\delta,r}$ be a composition operator, where $\delta$ is a
directional condition pair $d_1=v_1|v_2, d_2=v_1|v_2$ and $r$ is a
resolve function.  Then $\ttt_1 \odot_{\delta,r} \ttt_2 = \{ \tv',
\te', \tav', \tae' \}$, where $\forall (v_x,v_z,p) \in \te' \exists
(d_1,v_x,p_1) \in \te_1 \wedge (d_2,v_z,p_2) \in \te_2 \wedge p=p_1
\cap p_2$, \tv' contains only vertices with edges in \te', with FK
constraint enforced on \tav' from \tv', and $\forall (v_x,v_z,p,a) \in
\tae' \exists (d_1,v_x,p_1,a_1) \in \tae_1 \wedge (d_2,v_z,p_2) \in
\tae_2 \wedge p=p_1 \cap p_2 \wedge a=r((a_1,p_1),(a_2,p_2))$.
}

\eat{{\bf R1, R2}: coalesce $\te'$ and $\tae'$; {\bf R3}: constraint $\te'$
on $\tv'$, $\tv'$ on $\te'$ (to remove nodes with no edges), and
$\tae'$ on $\te'$; {\bf R4}: require reduce function.}


\subsection{Temporal User-Defined Analytics}
\label{sec:analytics}

For many types of evolving graph analysis, it is necessary to compute
some property, such as PageRank of each vertex $v$, or the length of
the shortest path from a given designated vertex $u$ to each $v$, for
each representative graph.  This information can then be used to study
how the graph evolves over time.  \ql supports this type of analysis
through {\em temporal user-defined analytics}, which conceptually
execute an analytic over each \rg, compute a value for each vertex of
the \rg, and then store this value among the attributes of the vertex
(adding it to the property bag).  We provide an API that allows
developers to implement custom analytics that can either be computed
locally at a vertex, like degree\eat{ (these are essentially map
functions~\ref{sec:algebra:map})}, or that can be expressed in the
popular Pregel API~\cite{DBLP:conf/sigmod/MalewiczABDHLC10}.

Logically, an analytic is computed on each representative graph.
However, we will show in Section~\ref{sec:exp} that more efficient
methods that use batching are possible.

%$udf(\trg) = \{ (g', p) | g' = udf(g) \cup g\}$

Analytics cannot be computed directly on \tve, because they must, by
definition, act over representative graphs.

{\bf Analytics do not uncoalesce.} An analytic adds a new key-value
pair to the property bag of each vertex $v$ of every representative
graph $g$, and does not implicitly project out vertex attributes.  For
this reason, an analytic cannot cause two temporally adjacent graphs
$g_1$ and $g_2$ that were not equivalent before the new attribute was
added to become equivalent after it was added.

\eat{With bag semantics, if the input was coalesced, then for any pair
  of consecutive $a_1, a_2$, $a_1 \neq a_2 \wedge a_1' \neq a_2'$.}

\eat{\vera{
2 options:
1) replace - compute the new values (vertex usually), return new graph with only that property. the user can then union back with the original graph to get old and new (this is how GraphX does it). this can uncoalesce.
2) append - compute the new values, add a new property to the bag with that value. this does not uncoalesce. option 2 described above.
}}



%\subsection{Examples}
\label{sec:example}

With the \ql operators in place, we can now show how the motivating
queries can be specified.

Question: In a sparse communication network, on what time scale can we
detect communities?

Process: a) compute new nodes with temporal windows of a set size; b)
compute connected components analytic; c) compute new nodes grouping
nodes by their component, computing their size; d) filter out nodes
that represent communities too small to be useful (e.g., of 1-2
people).  Examine the output and repeat from step a) with a different
temporal window size.

Query: $\sigma_{size > 2,true}($ \\ $ _{component}\vartheta_{1
  change,exists,exists,count(name),any}($ \\ $ pregel_{cc} (_v\vartheta_{6
  months,exists,always,first(name),any}( \ttt))))$

Question: Are there high influence nodes and is that behavior
persistent in time?

Process: a) compute in-degree (prominence) of each node for each time
interval it changed; b) compute new nodes with temporal window equal
to the overall timespan, collecting degree data into a map; c) apply
map to compute the coefficient of variation from the interval-degree
map.  Examine the output.

Query: $map_{stdev(degree)/mean(degree)*100}($ \\ 
$ _v\vartheta_{lifetime,exists,exists,map(degree),any}(
\gamma_{true,true,dst,1,count}( \ttt)))$

Question: How has graph in-degree centrality changed over time?

Process: a) compute new nodes with temporal windows of a set size; b)
compute in-degree of each node; c) compute new node grouping all nodes
co-existing in time into one, computing the sum, count, and max of
degree; d) apply map to compute the degree centrality.

Query: $map_{(max*count-sum)/(count^2-3count+2)}($ \\
$ _1\vartheta_{1 change,exists,exists,max(degree)\&sum(degree)\&count(degree),any}($ \\
$ \gamma_{true,true,dst,1,count}( _v\vartheta_{2 months,exists,always,any,any}( \ttt))))$

