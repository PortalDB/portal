\section{Introduction}
\label{sec:intro}

\vera{UNREVISED}

% General motivation
The importance of networks in scientific and commercial domains cannot
be overstated.  Considerable research and engineering effort is being
devoted to developing effective and efficient graph representations
and analytics.  Efficient graph abstractions and analytics for {\em
  static graphs} are available to researchers and practitioners in
scope of open source platforms such as Apache Giraph, Apache Spark /
GraphX~\cite{DBLP:conf/osdi/GonzalezXDCFS14} and GraphLab /
PowerGraph~\cite{DBLP:conf/osdi/GonzalezLGBG12}.

Arguably the most interesting and important questions one can ask
about networks have to do with their evolution, rather than with their
static state.  Analysis of {\em evolving graphs} has been receiving
increasing
attention~\cite{DBLP:journals/csur/AggarwalS14,Chan2008,Kan2009,Miao2015,Ren2011,Semertzidis2015}.
Yet, despite much recent activity, and despite increased variety and
availability of evolving graph data, systematic support for scalable
querying and analysis of evolving graphs still lacks.  This support is
urgently needed, due first and foremost to the scalability and
efficiency challenges inherent in evolving graph analysis, but also to
considerations of usability and ease of dissemination.  The goal of
our work is to fill this gap.  In this paper, we present an algebraic
query language called \tg algebra, or \tga, and its scalable
implementation in \ql, an open-source distributed framework on top of
Apache Spark.

Our goal in developing \tga is to give users an ability to concisely
express a wide range of common analysis tasks over evolving graphs,
while at the same time preserving inter-operability with temporal
relational algebra (\tra).  Implementing (non-temporal) graph querying
and analytics in an RDBMS has been receiving renewed
attention~\cite{DBLP:conf/sigmod/AbergerTOR16,DBLP:conf/sigmod/SunFSKHX15,DBLP:journals/pvldb/Xirogiannopoulos15},
and our work is in-line with this trend. Our data model is based on
the temporal relational model, and our algebra corresponds to temporal
relational algebra, but is designed specifically for evolving graphs.

\eat{We give a translation of our operators in relational algebra, and, as
a consequence, can implement our algebraic framework on top of a
relational engine.  Th advantages of incorporating graph querying
into a general-purpose relational system are two-fold: usability and
access to the relational data modeling and query processing
infrastructure.}

\eat{For this reason our data model is based on the temporal relational
model, and our query language corresponds to temporal relational
calculus, a two-sorted logic with variables and quantifiers explicitly
ranging over the time domain~\cite{DBLP:reference/db/Toman09}, for
graphs.}

We represent graph evolution, including changes in topology and in
attribute values of vertices and edges, using the \tg abstraction ---
a collection of temporal SQL relations with appropriate integrity
constraints.  An example of a \tg is given in Figure~\ref{fig:tg_ve},
where we show evolution of a co-authorship network.

\tga can be viewed as \tra for graphs, and so does not support general
recursion or transitive closure computation. (Although, as we will see
in Section~\ref{sec:analytics}, Pregel-style graph analytics such as
PageRank are supported as an extension.)  For this reason, we also do
not support regular path queries (RPQ) or the more general path query
classes (CRPQ and ECRPQ).  Extending our formalism with recursion and
path queries is non-trivial, and we leave this to future work.

Rather than focusing on path computation and graph traversal, we
stress the tasks that perform whole-graph analysis over time.  Several
such tasks are described next.  Additional examples can be found in
SocialScope~\cite{Amer-Yahia2009} --- a closed non-temporal algebra
for multigraphs that is motivated by information discovery in social
content sites.  It is not difficult to show that the graph (rather
than multigraph) versions of all SocialScope operations can be
expressed, and augmented with the temporal dimension, in \tga.

\section{Motivating use cases}
\label{sec:cases}

An interaction graph is one typical kind of an evolving graph.  It
captures people as graph verticesa, along with various information
about those people.  The edges are the interaction events between the
people, such as messages, conversations, tags, etc.  One easily
accessible interaction graph is the wiki-talk dataset
(\url{http://dx.doi.org/10.5281/zenodo.49561}) containing messaging
events among 3 million wiki-en users over a 13 year period.
Information available about the users includes their username, groups
they are part of, and their edit count, i.e. how many edits they have
produced on Wikipedia.  The messaging events occur when users post on
each other's talk pages.

We are primarily interested in analyses of the evolution of the
phenomena the graph represents.  However, simpler point queries should
be supported as well.

{\bf Example 1.}  In interaction graphs node centrality is a measure
of how important or influential people are.  Over a dozen different
centrality measures exist, providing indicators of how much
information ``flows'' through the vertex or how the vertex contributes
to the overall cohesiveness of the network.  The importance of nodes
fluctuates with time.  To see whether the wikitalk graph has high
prominence nodes and how stable importance is over some period of
time, we can extract the data for the period of interest, compute the
in-degree (or any other aggregated measure of centrality) for each
vertex and for each point in time and then calculate a coefficient of
variation.

This example demonstrates a need to select a subset of the data
corresponding to the period of interest, compute in-/out-degree for
each vertex at each point in time, and compute a single measure across
time for each vertex.

{\bf Example 2.}  Graph centrality is a popular measure to evaluate
how connected/centralized the community is.  Low centrality may
indicate that the community is disjointed or communicates poorly.  An
interesting measure by itself, it is subject to change as
communication patterns evolve or high influencers appear or disappear.
In sparse interaction graphs there is an additional question of what
time scale to consider: if two people communicated on May 16, 2010,
how long do we consider them connected?

This example demonstrates a need to compute graph centrality at every
point in its lifetime and to do so at different temporal resolution.
For most graph centrality measures, the two-step process involves
first calculating some measure, such as in-degree, for each graph
vertex, and then accumulating them into one.

{\bf Example 3.}  Interaction networks are sparse because the edges
are so short-lived.  To see whether communities form and at what time
scales, we can vary the time scale and compute communities,
e.g. through connected components detection, group the vertices by the
community they form and calculate their size.  We can filter out
vertices that represent communities below a reasonable threshold, for
example of size smaller than two.

This example demonstrates a need to compute graph-wide analytics such
as connected components for each point in time, create new vertices
that represent some aspect of data of existing vertices, and compute
subgraphs.

Besides the examples above, graph queries commonly include retrieving
a specific node (which can be accomplished through a subgraph), k-hop
neighborhood of a node, and local transformations of attributes of a
node or edge.  If multiple sources of the same graph are available, it
is useful to combine or compare them, which dictates the need for
regular set-theoretic operators.

In Section~\ref{sec:algebra} we define the operators of our graph
algebra and show how our use cases can be expressed using these
operators.


\subsection{Contributions and roadmap}

%
%\begin{enumerate}[noitemsep,leftmargin=*]
%\begin{description}[noitemsep]
%
 We propose a representation of an evolving graph, called a \tg, which
 captures the evolution of both graph topology and vertex and edge
 attributes (Section~\ref{sec:model}), and develop a compositional \tg
 algebra, \tga (Section~\ref{sec:algebra}).
%
We show a reduction from \tga to temporal relational algebra
  \tra, using a combination of standard operators and \tg-specific
  primitives, and present formal properties of \tga
  (Section~\ref{sec:formal}).
%
We present an implementation in scope of the \ql system, built
  on Apache Spark / GraphX~\cite{DBLP:conf/osdi/GonzalezXDCFS14}. \ql
  supports several access methods that correspond to different
  trade-offs in temporal and structural locality
  (Section~\ref{sec:sys}).
%
We conduct an extensive experimental evaluation with real datasets,
demonstrating that \ql scales (Section~\ref{sec:exp}).  We also
illustrate the usability throughout the paper, with a variety of
real-life analysis tasks that can be concisely expressed in \tga.

%\end{enumerate}


