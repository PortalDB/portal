\begin{table}
\centering
\caption{Connected components and PageRank with different temporal partitioning, seconds.}
\vspace{-0.2cm}
\begin{subtable}{0.229\textwidth}
\small
\caption{wiki-talk}
\vspace{-0.2cm}
\begin{tabular}{| l | c | c | c |}
\hline
\multicolumn{1}{|l|}{\bfseries width} & \multicolumn{1}{c|}{\bfseries k} & \multicolumn{1}{c|}{\bfseries CC} & \multicolumn{1}{c|}{\bfseries PR} \\ \hline
8 & 23 & 382 & 1,086 \\ \hline
16 & 12 & 328 & 1,037 \\ \hline
bal. & 16 & 351 & 720 \\ \hline
bal. & 24 & 408 & 375 \\ \hline
\end{tabular}
\label{fig:splitwiki}
\end{subtable}
\begin{subtable}{0.229\textwidth}
\small
\caption{nGrams}
\vspace{-0.2cm}
\begin{tabular}{| l | c | c | c |}
\hline
\multicolumn{1}{|l|}{\bfseries width} & \multicolumn{1}{c|}{\bfseries k} & \multicolumn{1}{c|}{\bfseries CC} & \multicolumn{1}{c|}{\bfseries PR} \\ \hline
8 & 26 & 1,324 & 2,867 \\ \hline
16 & 13 & 856 & 1,873  \\ \hline
bal. & 3 & 321 & 740  \\ \hline
bal. & 16 & 422 & 519 \\ \hline
\end{tabular}
\label{fig:splitngrams}
\end{subtable}
\label{tab:splitres}
\vspace{-0.4cm}
\end{table}

%\begin{figure*}
%\begin{subfigure}{0.45\textwidth}
%\includegraphics[width=3.4in]{figs/splitters_wiki.pdf}
%\label{fig:splitwiki}
%\caption{wiki-talk}
%\end{subfigure}
%\begin{subfigure}{0.45\textwidth}
%\includegraphics[width=3.4in]{figs/splitters_ngrams.pdf}
%\label{fig:splitngrams}
%\caption{nGrams}
%\end{subfigure}
%\caption[]{Connected components and PageRank with different temporal partitioning.}
%\label{fig:splitres}
%\end{figure*}

{\bf Preliminary experiments.}  We conducted some preliminary
experiments to see the effect of temporal partitioning on distributed
execution of analytics, which present one of the heaviest
computational workloads.  PageRank and Connected components analytics
were executed on the wiki-talk\footnote{\url{http://dx.doi.org/10.5281/zenodo.49561}} and nGrams\footnote{\url{http://storage.googleapis.com/books/ngrams/books/datasetsv2.html}}
datasets.

Wiki-talk contains 179 time periods.  nGrams contains over 400, but we
used the first 208.  Both datasets exhibit strong skew, with few edges
at the start of the datasets and increasing by several orders of
magnitude towards the end.  We compared equi-width and equi-depth
temporal partitioning, using 8 and 16 consecutive intervals for
equi-width, and using offline optimal split of edges with varying
number of splitters $k$.  Each dataset was partitioned first
temporally, and then spatially using Edge2D partitioning.
Table~\ref{tab:splitres} shows that equi-depth partitioning is
superior to equi-width in all but one cases.  However, the number of
splitters is key in obtaining good results.  We are currently
investigating this phenomenon further.
