\section{Related work}
\label{sec:related}

{\bf Evolving graph models.}  While temporal models in the relational
literature are very mature, the same cannot be said about the evolving
graphs literature.  Evolving graph models differ in what time stamp
they use (point or interval stamping), what top-level entities they
model (graphs or sets of nodes and edges), whether they represent
topology only or attributes or weights as well, and what types of
evolution are allowed.  All evolving graph models require node
identity, and thus edge identity as well, to persist across time.
See~\cite{Zaki2016} for a survey of evolving graph models.

The first mention of evolving graphs that we are aware of is by Harary
and Gupta~\cite{Harary1997} who informally proposed to model the
evolution as a sequence of static graphs.  This model has been
predominant in the research literature
(\cite{Fard2012,Khurana2013,Ren2011}
and many others), with various restrictions on the kinds of changes
that can take place during graph evolution.  For example, Khurana and
Deshponde use this model with the restriction that a node, once
removed, cannot reappear~\cite{Khurana2013}.  In~\cite{Fard2012}
and~\cite{Ren2011} there is no notion of time, only a sequence of
graphs.  \eat{In~\cite{Borgwardt2006} and~\cite{Kan2009} the time
  series of graphs represents topology only, with no attributes, and
  only edges can vary with time, while the nodes remain unchanged.
  Ferreira's model~\cite{Ferreira2004} allows both node and edge
  evolution, but again, only restricted to topology.  }It is important
to note that we are talking about the logical model of the evolving
graphs, rather than a physical representation.  For example,
Semertzidis et al. present a concrete representation of evolving
graphs called VersionGraph which is similar to the logical model we
propose here~\cite{Semertzidis2015}, but even there the logical model
is still a sequence of graph snapshots.

The advantages of the snapshot sequence model are that (a) it is
simple and (b) if snapshots are obtained by periodic sampling, which
is a very common approach, it accurately represents the state at that
point without making assertions about unknown times.  For example, the
WWW is so large that it is impossible to create a fully accurate
snapshot that represents any moment in time.  An important limitation
of this model, in addition to the semantic ambiguity on what
constitutes a change, is that it forces a specific time granularity,
whereas open-closed time intervals can be broken down into any desired
level of granularity.

\eat{Several models use the interval time model, although without the full
sequenced semantics.  Huo and Tsotras model an evolving graph as a set
of nodes and a set of edges where each entity is
interval-stamped~\cite{Huo2014}.  The semantics of the intervals is
not stated.  Similarly, Koloniari~\cite{Koloniari2012}, while not
providing a formal model, represents nodes and edges as having periods
of validity.  As~\cite{Bohlen1998} has shown, sequenced semantics is
determined now by the representation, but by the properties of the
operators.}

\eat{There are several important system works on evolving graphs which do
not have a formal model.  The ImmortalGraph (formally Chronos) system
has no formal model, but informally an evolving graph is defined as a
series of activities on the graph, such as node additions and
deletions~\cite{Miao2015}.  This is a streaming or delta approach,
which is popular in temporal databases because it is unambiguous and
compact.  The G* system also does not propose a formal model for
evolving graphs, even though it can be used for storing and analyzing
them by maintaining periodic snapshots~\cite{Labouseur2015}.  So while
there is no explicit notion of time in G*, it essentially uses
discrete time snapshot sequence model, with all limitations thereof.
For each graph, it adopts a nested data model, with vertices nested
within the graph, and edges within the corresponding vertices.}

\eat{This situation has persisted over 20 years, due in large part to the
fact that most papers lack formal semantics of the model or the
operations, which are generally specified algorithmically.}

{\bf Temporal relational models.}  The question of semantics of
temporal data has been thoroughly explored in the relational temporal
database community.  B{\"{o}}hlen et al.~\cite{Bohlen1998} defined
point and sequenced models, and showed that the difference between the
model lies in the properties of the operators, and not in the use of
intervals as representational devices.  With this foundation,
Dign{\"{o}}s et al.~\cite{Dignos2012} defined sequenced semantics,
with properties of snapshot reducibility, extended snapshot
reducibility, and change preservation.\eat{, and showed how
  non-temporal operators can be applied to temporal data. } Snapshot
reducibility means that a temporal operator produces the same result
as an equivalent non-temporal operator over corresponding snapshots.
Extended snapshot reducibility allows references to timestamps in the
operators by propagating them as data.  Point semantics has both of
these properties as well.

The third property, change preservation is unique to sequenced
semantics.  It states that operators only merge contiguous time points
of a result if they have the same lineage.  As shown
in~\cite{Dignos2012}, all three properties can be guaranteed through
the use of the normalize and align operators to non-temporal relations
extended with the explicit time attribute.  The normalize operator
splits each tuple in the input relation w.r.t. a group of tuples such
that each timestamp fragment is either fully contained or disjoint
with every timestamp in the group.  The align operator splits each
tuple w.r.t. a group of tuples such that each timestamp fragment is
either an intersection with one of the tuples in the group or is not
covered by any tuple in a group. (See Figure 2 in~\cite{Dignos2012}
for an illustration.)

As we as a community move to more and more sophisticated analyses of
evolving graphs, we need to adopt the state of the art in temporal
databases.
