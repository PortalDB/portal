\section{Algebra}
\label{sec:algebra}

In this section we describe the operators of \tg algebra.  The algebra
is compositional --- all operators take a \tg or a pair of \tgs as
input, and produce a valid \tg as output.

\eat{ \tg algebra opertors come in two flavors.  The first are
  operations defined directly on the \rg representation, namely,
  analytics and temporal aggregation by change.  The second are
  operations such as selection and projection, which concern the
  structure of the elements of each representative graph --- its
  vertices, edges, and attribute values.  Some operators can be
  expressed (and computed) on both representations, and for such
  operators we give both semantics, and show that the result is
  equivalent.  Some operators, including \tg union, join and
  difference.... }

\subsection{Preliminaries}
\label{sec:algebra:prelim}

B\"ohlen et al.~\cite{DBLP:conf/vldb/BohlenSS96} show that temporal
selection, Cartesian product and difference all produce a coalesced
relation as output if the input was coalesced.  They also show that
temporal union and temporal projection can give rise to an uncoalesced
output even if the inputs were coalesced.  Intuitively, this is
because union and projection can give rise to duplicates in
traditional relational algebra, and lack of coalescing is the temporal
analogy to a duplicate.

These observations also hold in our scenario at the level of
individual relations, each containing representative graphs, vertices,
edges, or vertex/edge attributes.

Our algebra operates on \tgs, and so, in addition to keeping the data
structure (and its constituent parts) coalesced, we must ensure that
the result is a valid \tg.\eat{ This objective informed our design of
  the algebra, e.g., we did not include some flavors of temporal
  aggregation because the result would be invalid.  Further, this
  objective informs the rewriting of applicable operations over the
  \ve representation.}  Importantly, when manipulating the \ve
representation, we do not require that each intermediate state of the
data structure correspond to a valid \tg, but rather that the final
result, which is usually derived after several steps, be valid.  To
have a useful algebra, we do not reject a change that would lead to a
violation in referential integrity.  Instead, we compute a consistent
result by removing the tuples that violate referential integrity.

Consider a unary operation \op and suppose that it is being evaluated
over the \ve representation \op(\tve).  Algorithm~\ref{alg:op}
outlines the steps in the evaluation, overloading \opp as appropriate
when applying the operation to constituent parts of \tve.  Some of the
steps in Algorithm~\ref{alg:op} may be unnecessary because of the
properties of the particular operation, as we will see in the
remainder of this section.  Furthermore, some of the operations may
produce correct results (up to coalescing) {\em even when computing
  over uncoalesced inputs}.  We will revisit this point when
discussing lazy evaluation in Section~\ref{sec:sys}.

\begin{algorithm}[h!]
\caption{Evaluation of a unary operation on \tve}
\begin{algorithmic}[1]
\REQUIRE \tg $\tve (\tv; \te; \tav; \tae)$, operation \insql{op}.\\
\STATE  $\tv' = \cl (\opp(\tv))$\\
\STATE  $\te' = \cl (\opp(\te))$\\
\STATE  $\tav' = \cl (\opp(\tav))$\\
\STATE  $\tae' = \cl (\opp(\tae))$\\
\STATE  enforce foreign keys on $\te'$ w.r.t. $\tv'$\\
\STATE  enforce foreign keys on $\tav'$ w.r.t. $\tv'$\\
\STATE  enforce foreign keys on $\tae'$ w.r.t. $\te'$\\
\RETURN new $\tve (\tv', \te', \tav', \tae')$\\
\end{algorithmic}
\label{alg:op}
\end{algorithm}

\subsection{Slice}
\label{sec:algebra:slice}

The unary {\em slice} operator, denoted $\xi_c (T)$, where $c$ is a
time period, cuts a temporal slice from $T$.  The resulting \tg will
contain representative graphs whose period $p$ has a non-empty
intersection with $c$ (i.e., $p$ is either contained within $c$ or
overlaps with it, see Section~\ref{sec:model:prelim}).  If $p.start <
c.start$ or $p.end > c.end$ for some tuple $(g, p)$, then $p$ is
trimmed to be within the boundaries of $c$: $\xi_c (\trg) = \{ (g, p
\cap c)~~|~~(g, p) \in \trg \wedge (\pred{c}{overlaps}{p} \vee
\pred{c}{contains}{p})\}$.

To evaluate $\xi_c (\tve)$, we apply $\xi_c$ to each of the four
constituent relations of \tve.  For example: 
$\xi_c (\tv) = \{ (v, p
\cap c)~~|~~(v, p) \in \tv \wedge (\pred{c}{overlaps}{p} \vee \pred{c}{contains}{p}) \}$, 
and analogously for each \te, \tav and \tae.

\eat{ 
In SQL, slice can be expressed as follows for $V$ (similarly for other
relations):}

\eat{\begin{small}
\begin{verbatim}
SELECT vid, a1, ..., an, greatest(estart, DATE ':date1'), 
       least(eend, DATE ':date2')
FROM V
WHERE eperiod OVERLAPS PERIOD (DATE ':date1', DATE ':date2')
\end{verbatim}
\end{small}}

{\bf Slice does not uncoalesce.} Whether evaluated over \trg or \tve,
slice is guaranteed to return a coalesced relation when evaluated over
a coalesced input.  This is because, for any relation \insql{R}, there
will be at most one tuple from \insql{R} in the result of $\xi_c (R)$,
with a validity period that is either the same as it was in \insql{R},
or further restricted (trimmed).  Therefore, there is no need to
coalesce \trg after slice, or on lines 1-4 of Algorithm~\ref{alg:op}
when operating over \tve.

{\bf Slice does not require FK enforcement for \tve.}  To see why,
consider an edge $\te(v_1, v_2, p)$ and one of the corresponding
vertices $\tv(v_1, p_1)$, such that $\pred{p_1}{contains}{p}$ (per
Definition~\ref{def:tg} condition~\ref{def:tg:c1}).  Suppose now that
slice was applied to \tv and to \te with condition $c$.  Is it
possible that edge $(v_1, v_2, p \cap c)$ is in the result of $\xi_c
(\te)$ (i.e., $p \cap c \neq \emptyset$), while vertex $\tv(v_1, p_1 \cap
c)$ is not in the result of $\xi_c (\tv)$ (i.e., $p_1 \cap c =
\emptyset$)?  Clearly, the answer is no, since
$\pred{p_1}{contains}{p}$, and so it must be the case that
$\pred{p_1}{contains}{(p \cap c)}$.  A similar argument justifies that
FK enforcement is not needed for $\xi_c(\tav)$ (w.r.t. $\xi_c (\tv)$) and
for $\xi_c(\tae)$ (w.r.t. $\xi_c (\te)$).

\subsection{Temporal subgraph matching}
\label{sec:algebra:subgraph}

Temporal subgraph matching is defined analogously to subgraph matching
in non-temporal graphs: it applies a subgraph function $f$ to every
representative graph of the input.  To ensure that a valid \tg is
computed as a result of this operation, we restrict our attention to
functions that compute a single subgraph of a given representative
graph as a result: $\sigma_f (\trg) = \{ (g', p)~~|~~(g, p) \in \trg
\wedge g' = f(g) \wedge$\\$V_{g'} \subseteq V_{g} \wedge E_{g'}
\subseteq E_{g} \}$.

\eat{For example, $f$ may select a subset of the vertices or edges of
  $g$ based on some condition that can be computed locally at a vertex
  or edge, or by a path / reachability expression.  In this paper we
  restrict our attention to conjunctive queries.}  Even more
specifically, we focus on functions those that can be expressed as a
pair of {\em conjunctive queries} $Q_V$, with {\em non-temporal
  predicates} over the vertices of $g$, and $Q_E$, with {\em
  non-temporal predicates} over the edges of $g$. (Computing arbitrary
subgraphs of an evolving graph is beyond the scope of this paper, and
warrants a deeper investigation, which we defer to future work.)

Like other unary operators, {\em temporal subgraph matching}
$\sigma_{Q_V, Q_E} (\tve)$ follows the outline of
Algorithm~\ref{alg:op}.  Importantly, since $Q_V$ and $Q_E$ may
involve predicates over the attributes, we compute the join of the
vertex (resp. edge) relation with the corresponding attribute relation
to evalutate the query, and push selections as appropriate.  Line 1 of
Algorithm~\ref{alg:op} becomes: $V' = \pi_{v,p} (\sigma_{Q_{V1}} (V)
\bowtie \sigma_{Q_{V2}} (\tav))$.  Similarly, $E' = \pi_{v_1,v_2,p}
(\sigma_{Q_{E1}} (E) \bowtie \sigma_{Q_{E2}} (\tae))$ (line 2 of
Algorithm~\ref{alg:op}).  We compute $\tav' = \sigma_{Q_{V2}} (\tav)$
(line 3) and $\tae' = \sigma_{Q_{V2}} (\tae)$ (line 4).

{\bf Subgraph does not uncoalesce \tve.}% but may uncoalesce \trg.}
Consider again the computation of $V'$ described above, with a query
that involves projection, selection and join over temporal SQL
relations $V$ and \tav.  While selection and join cannot produce an
uncoalesced output if the input is coalesced, this is generally not
the case for projection~\cite{DBLP:conf/vldb/BohlenSS96}.
Interestingly, projection also does not result in an uncoalesced
output in this case. To see why, suppose that $Q_V$ is trivial, i.e.,
that $\sigma_{Q_{V1}} (V) = V$ and $\sigma_{Q_{V2}} (\tav) =
\tav$. Then $V' = \pi_{v,p} (V \bowtie \tav)$, and since $V \bowtie
\tav$ is a primary key-foreign key join, then $V' = V$.  If $Q_V$ is
non-trivial, i.e., $\sigma_{Q_{V1}} (V) \subset V$ or
$\sigma_{Q_{V2}} (\tav) \subset \tav$, then it will be the case that
$V' \subset V$.  In both cases, if $V$ is coalesced then so is $V'$.
A similar argument applies to the edges relation $E'$.  Finally, since
\tav' and \tae' are computed from coalesced input relations using
selection, these relations are guaranteed to be coalesced.  Thus, it
is not necessary to coalesce on lines 1-4 of Algorithm~\ref{alg:op}.

{\bf Subgraph does require FK enforcement for \tve.}  Perhaps the most
natural temporal subgraph query is one that specifies a selection
condition over the vertices, and computes the vertex-induced subgraph.
In this case we cannot compute $E'$ from $E$ alone, but will also need
to remove edges for which one or both vertices are not present in $V'$
during the specified time period.  Similarly, we need to remove tuples
from $\tav'$ and $\tae'$ for which no corresponding tuples exist in
$V'$ and $E'$, respectively.

{\bf Subgraph may uncoalesce \trg.} Consider the result of
$\sigma_{Q_V:{a.school='Drexel'},Q_E:\top} (\insql{T1})$ when
applied to the \tg in Figure~\ref{fig:rg}.  This query keeps
vertices 1 and 3 in every representative graph, and no edges.  Since
graphs corresponding to time periods $p1$ through $p4$ will be
identical, the result will be uncoalesced, and will need to be
coalesced explicitly.  The final result will consist of 2
representative graphs, with vertices 1 and 3 for $[t_0, t_4)$ and with
  vertex 3 for $[t_4, t_5)$.

\subsection{Temporal projection}%; vertex- and edge-map}
\label{sec:algebra:project}

\tg algebra supports a limited kind of projection: while the key
attributes of vertices and edges must be retained, it is possible to
project out some or all non-key vertex and edge attributes.
$\pi_{\avv',\aee'} (\trg) = \{ (g', p)~~|~~(g, p) \in \trg \wedge
V_{g'} = V_{g} \wedge E_{g'} = E_{g} \wedge \avv' \subseteq \avv
\wedge \aee' \subseteq \aee \}$.  To evaluate projection over \tve, we
compute $\tav' = \pi_{\avv'} (\tav)$ and $\tae' = \pi_{\aee'} (\tae)$.

Because \avv and \aee are nested relations, projection of the kind
defined here can be taken more broadly to mean {\em map}.  The user
may specify an arbitrary map function that is applied to each tuple in
\avv (resp. \aee), and transforms the properties of vertex $(v,p)$
(resp. of edge $(v_1,v_2,p)$).  In addition to removing properties, it
will often be useful to, e.g., flatten nested collections or 
reconcile multiple values of the same property, e.g., compute a sum or
an average when map is invoked following temporal aggregation
(Section~\ref{sec:algebra:agg}), or temporal intersection or union
(Section~\ref{sec:algebra:join}).

{\bf Projection / map may uncoalesce \tav, \tae and \trg.}  Consider
\insql{T1} in Figure~\ref{fig:ve}, and suppose that we are computing
$\pi_{a:(name)} (\tav)$.  There will be two identical tuples in the
result for vertex $v_2$ for $[t_1, t_2)$ and $[t_2, t_5)$, which must
    be coalesced to return a valid \tav.  A similar argument holds for
    \tae. This operation will also produce two identical
    representative graphs in \trg in Figure~\ref{fig:rg} for $[t_1,
      t_2)$ and $[t_2, t_5)$, which will have to be coalesced
        explicitly.

{\bf Projection and map do not require FK enforcement for \tve.}  This
is because the only relations that are affected by this operation are
\tav and \tae, while the contents of \tv and \te remain as in the
input.

\subsection{Temporal aggregation}
\label{sec:algebra:agg}

We argued in the introduction that it is interesting and insightful to
analyze an evolving graph at different levels of granularity.  For
example, the user may want to aggregate multiple consecutive
representative graphs into a single representative graph, coarsening
the granularity, or to pre-define temporal resolution and look at the
graph at that scale, irrespective of whether this resolution happens
to be finer or coarse than the natural evolution rate of the graph.
For this, we will use temporal aggregation.  Our approach is inspired
by stream aggregation work by Li, et al.~\cite{Li2005}, adopted to
graphs, and by generalized quantifiers of~\cite{Hsu1995}.

The {\em temporal aggregation} operator is denoted $\gamma_{w,q}
(\trg)$, where $w$ is the window specification and $q$ is the
aggregation quantifier.  

{\em Window specification} is of the form
$n~\{unit|\insql{changes}\}$, where $n$ is an integer, and $unit$ is a
time unit, e.g., ``minutes'' or ``days''.  Specifying, e.g.,
$10~minutes$ defines the window in terms of time, e.g., a 10-minute
window or, more generally any multiple of the usual time units
(minutes, hours, days, weeks, months, years).  Window specification of
the form $n~\insql{changes}$ defines the window in terms of change,
e.g., aggregate sequences of 3 representative graphs into 1.  Window
boundaries are computed left-to-right, starting at the start of the
least recent represenatative graph.

Our window specificaiton is similar to slide-by-row window in stream
aggregation~\cite{Li2005}.  Note that, because \tg algebra is
compositional, we do not support temporal aggregation with overlapping
windows. Also unlike~\cite{Li2005} we do not currently support
aggregation simultaneously by time and by non-temporal attributes
(e.g., vertex attributes). Incorporating this into \tg algebra and
system implementation is in our immediate plans.

{\em Aggregation quantifiers} are of the form \{at\ least\ one | all |
most | at least $n$ \}, where $n$ is a decimal representing a ratio.
      These are useful for producing different kinds of representative
      graphs.  For example, to produce representative graphs with only
      strong connections over a volatile evolving graph, we may want
      to only include edges that span the entire aggregtion window, or
      a large portion of the window.  

Note that a particular entity (representative graph, vertex or edge)
may be mapped to one or multiple aggregation windows.  Consider for
example

\eat{ Our aggregation quantifiers are inspired by generalized
  quantifiers of~\cite{Hsu1995} with n-place delimiters.  $Q(R)$ as a
  Boolean-valued function of a relation''~\cite{Hsu1995}.  A
  quantifier contains an n-place determiner, e.g., ``at least one
  vertex in each window for each group'' is a 2-place determiner
  quantifier.  \tg algebra supports determiners from the set
  $\{at\ least\ one, all, most, at\ least\ n\}$, where $n$ is an
  integer representing a ratio.  $all$ is a usual universal quantifier
  that in standard SQL can be achieved with the use of two \insql{NOT
    EXISTS}.}

\eat{Aggregation in relational algebra produces results for each grouping
irrespective of how many results there are, unless a \insql{HAVING}
restriction is applied.  The quantification over the aggregation
results in evolving graphs is useful for producing different kinds of
representative graphs.  For example, to produce a representative graph
with only strong connections over a volatile evolving graph, we want
to restrict results to those edges that span the entire window or a
large subset of that window.  For this purpose we introduce
quantifiers.}

%%%%%%%%%%%%%%%%%%%%%%%%%%%%%%%%%%%%%%%%%%%%%%%%%%%%%%%%%%%%%%%%%%%%

\eat{
The second observation is that it is necessary to enforce referential
integrity.  Consider a pair of relations $R_i, R_j \in T_{VE}$, such
that there is a foreign key on $R_j$ referencing $R_i$, and consider
the couterparts of these relations $R'_i = \sigma_{c_i} R_i$ and $R'_j
= \sigma_{c_j} R_j$.  If the selection condition on $R_i$ is trivial
(i.e., $R_i = \sigma_{c_i} R_i$), then referential integrity will hold
on $R_i, R_j$.  However, if $c_i$ removes some tuples from $R_i$, then
it becomes necessary to idenfity tuples in $R'_j$ for which there is
no counterpart in $R'_i$ and delete them.}

\eat{
Recall that a \tg is a pair of coalesced temporal SQL relations, with
an integrity constraint that ensures that an edge exists at a time
when both vertices it connects also exist.  We start by investigating
the behavior of our model under a standard definition of temporal
relational algebra operators applied to $V$, $E$ or both in
Section~\ref{sec:algebra:rel}.  We then present the novel operations
of \tg algebra (Section~\ref{sec:algebra:graph}).  The main challenge
in both sub-section is understanding whether and when to coalesce $V$
and $E$, and how to efficiently enforce the integrity of the data
structure.}

%\subsection{Relational algebra operators over $V$ and $E$}

\eat{$V$ and $E$ are valid-time temporal relations, and we adopt (and
adapt) the semantics of period-based temporal
algebra~\cite{DBLP:conf/vldb/BohlenSS96} to our setting. }

\eat{1) Apply operation to V
2) Apply operation to E
3) Coalesce V if necessary
4) Coalesce E is necessary
5) Enforce integrity constraint}

\eat{Temporal selection $\sigma_c V = \{ \langle v, p, a_1, \ldots, a_n
\rangle | c(\langle v, p, a_1, \ldots, a_n \rangle) \}$ returns a
subset of the tuples in $V$. Note that the selection condition $c$ is
an arbitrary boolean condition that may also include predicates on
$p$.  }

\eat{When evaluated over coalesced input relations, temporal selection,
temporal Cartesian product and temporal negation preserve coalescing.}

\eat{
\begin{definition}[Selection]
Temporal and structural selection on $TG$ is a selection on the
attributes of $V$ and $E$, including entity periods.  $\sigma_{a
  \theta c}(TG)$, where $a$ are attributes of $V$ and/or $E$,
including periods, $\theta$ is a binary operation in the set $\{<,
\leq, =, \neq, \geq, >\}$, and $c$ is a value constant.
\label{def:selection}
\end{definition}}

\eat{Temporal and structural selection are supported by the same selection
operator and can be used together.  For example, one could select a
sub-graph in an evolving co-citation network of only authors whose
names start with letter A, over the past decade.  Because of the
constraint on $E$, even if the structural selection is only on
attributes of $V$, only edges connecting selected vertices are
retained.  Neither deduplication nor coalescing is required as a
post-operation.  Note that temporal selection and slice are different
because temporal selection does not modify entity periods, only
selects some of them.}

\eat{In SQL, selection can be expressed as a regular selection on V,
followed by a selection on E with integrity constraint enforced.}

\eat{\begin{definition}[Projection]
Projection on $TG$ is projection on attributes of $V$ and $E$ with
coalescing, i.e. \\$\Pi vid, p, a_1, \ldots, a_n(V); \Pi vid_1, vid_2,
p, b_1, \ldots, b_m(E)$. 
\label{def:projection}
\end{definition}}

\eat{\begin{definition}[Slice]
The unary operation \op{slice}, denoted $\sigma_{[start, end)}
  \insql{T}$ is a selection operation that includes...}

\eat{Slice on $TG$ is a selection on periods of $V$ and $E$ such that
$slice_{[a,b)}(TG) = \{t': t \in TG$, $t(p).overlaps(period(a,b)), t'
  = fit(t, period(a,b))\}$ and $fit(t, period(a,b))$ shortens the
  entity period $p$ to be within $[a,b)$.
\label{def:slice}
\end{definition}}

\eat{In SQL, slice can be expressed as follows for $V$
(similarly for $E$):}

\eat{\begin{small}
\begin{verbatim}
SELECT vid, a1, ..., an, greatest(estart, DATE ':date1'), 
       least(eend, DATE ':date2')
FROM V
WHERE eperiod OVERLAPS PERIOD (DATE ':date1', DATE ':date2')
\end{verbatim}
\end{small}}

%\subsection{Temporal aggregation}

\eat{Now that we have the window semantics and quantification defined, we
can define the aggregation operation over an evolving graph $TG$.}

\eat{It is often useful to analyze aggregate behavior of an evolving graph
over some coarser time period.  For example, a union of all
vertices/edges over 1 month is representative of that graph during
that month.  Analysis of aggregate behavior can lead to deeper insight
than of a snapshot.  For example, a co-authorship network DBLP is very
sparse --- one can only publish so many papers in any given month ---
on a daily or even monthly level of granularity, but can show
community formation and affiliation when aggregated over multiples of
years.  From this perspective, an evolving graph is a sequence of
representative graphs over consecutive arbitrary-length periods.}

\eat{
\begin{definition}[\tg Aggregation]
An {\em aggregation} operation over $TG$ is a function \\ $G_1, G_2,
\ldots, G_n, W, Q g f_1(A_1), f_2(A_2), \ldots, f_m(A_m)(TG)$, where
each $G_i$ is a grouping attribute from $TG$ with the exception of
$p$; $W$ is the window specification; $Q$ is a generalized quantifier
specification for vertices and edges on the coverage of the window;
each $F_i$ is an aggregation function; and each $A_i$ is an attribute
name from $TG$.
\label{def:agg}
\end{definition}}

This definition is similar to the regular relational algebra
aggregation definition, with the addition of the window specification
and the restriction of grouping attributes to exclude the time
periods.  However, both $V$ and $E$ relations are aggregated by the
same operation and the constraint on $E$ to contain only those
vertices that exist in $V$ is maintained.  The aggregation defined
this way allows to aggregate graphs structurally, temporally, or in
combination.  To aggregate only temporally, the grouping attribute
must be $vid$ for $V$ and $(vid1, vid2)$ for $E$.  To aggregate only
structurally, the window specification must be by 1 change.  Observe
that aggregation by 1 change with grouping by id is a no-op, and in
fact the sequence of representative graphs on the source data is equal
to the deduplicated sequence of snapshots.

The quantifier is applied to the coverage of the window period within
each grouping.  For example, to construct the persistent edges graph
from the example above, we use the $all$ quantifier over the $E$ relation.
Only the edges that span the duration of the window period are
produced.  Since the universal quantification is very restrictive,
$most$ and $at\ least\ n$ quantifiers are more appropriate in some
aggregations, especially over long windows.  To obtain a stable
1-month graph over an evolving network connections graph, we may ask
for connections that exist in at least 90\% of the period.

Remember that the schema for $V$ has a $(vid, p)$ primary key.  Any
aggregation operation must produce both a valid $vid$ and valid time
interval for each tuple.  We produce a $vid$ by using the hash of the
grouping variables to maintain temporal consistency.  The time
interval is produced by the window extent from the window
specification. Similarly for $E$.

The aggregation functions over the selected graph attributes are used
to compute the new value representative of the whole window.  We
support the standard set $\{count, min,$ $max, sum, average\}$, as
well as $\{any, first, last, list\}$.  $first$ and $last$ refer to
first/last non-null value in the window and are possible because the
aggregating tuples have the time dimension, and the ties are decided
arbitrarily.  $count$ is the count of the number of distinct values
over the aggregation window.  Additional aggregation functions can be
defined by the user.

\begin{figure*}
\includegraphics[width=6.5in]{figs/agg1.pdf}
\caption{1-month window aggregation with grouping by id, with
  existential vertex/edge quantifier.  Structure only.}
\label{fig:agg1}
\end{figure*}

An example in Figure~\ref{fig:agg1} shows a small \tg and a result of
aggregation by 1 month on that graph with $at\ least\ one$ quantifier,
with group by id.

It is possible but difficult to express the aggregation operation as
defined above in temporal SQL because each tuple may belong to
multiple windows.

\subsection{Temporal union and intersection}
\label{sec:algebra:join}

\begin{definition}[Union]
Union of $TG1\ \cup\ TG2$ = $\{v, e: v \in TG1.V$ or $v \in TG2.V$ or
$v(vid$, $f_1(a_{11}$, $a_{21})$, \ldots, $f_n(a_{1n}, a_{2n})$,
$period(least(p_1, p_2), greatest(p_1, p_2)))$, $e \in TG1.E$ or $e
\in TG2.E$ or $e(vid1, vid2, g_1(b_{11}, b_{21}), \ldots, g_m(b_{1m},
b_{2m})$, $period(least(p_1, p_2), greatest(p_1, p_2))) \}$, where
each $f$, respectively $G$, is an aggregation function over one
vertex, respectively edge, attribute where the values intersect over
some period $p$.
\label{def:union}
\end{definition}

\begin{figure*}
\includegraphics[width=6.5in]{figs/union.pdf}
\caption{Union of TG1 and TG2.}
\label{fig:union}
\end{figure*}

In other words, union of two \tgs is simply all vertices and edges
from both \tgs, with vertex/edge attributes decided by specified
aggregation functions for each period where both values are present,
with coalescing.  Figure~\ref{fig:union} illustrates this concept.

Similarly, intersection of two \tgs is an intersection of
vertices/edges of two graphs, with values for each overlapping
attribute computed by a specified aggregate function.

