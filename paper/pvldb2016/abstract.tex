\begin{abstract}

Graphs are used to represent a plethora of phenomena, from the Web and
social networks, to biological pathways, to semantic knowledge
bases. Arguably the most interesting and important questions one can
ask about graphs have to do with their evolution. Which Web pages are
showing an increasing popularity trend? How does influence propagate
in social networks? How does knowledge evolve?

Much research and engineering effort today goes into developing
sophisticated graph analytics and their efficient implementations,
both stand-alone and in scope of data processing platforms. Yet, {\em
  systematic support} for scalable querying and analytics over {\em
  evolving graphs} still lacks.

In this paper we present \ql, a system that supports efficient
exploratory analysis of evolving graphs. \ql is implemented in scope
of Apache Spark, an open-source distributed data processing framework,
and supports a variety of operations over evolving graphs, including
temporal and structural selection, aggregation, and a rich class of
analytics.  We develop multiple physical representations of evolving
graphs and study the trade-offs between structural and temporal
locality.  We provide an extensive experimental evaluation on real
datasets, demonstrating that careful engineering can lead to good
performance.

\end{abstract}
