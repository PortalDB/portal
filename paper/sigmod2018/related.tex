\section{Related work}
\label{sec:related}

Structural zoom out is supported in some nontemporal graph systems
through a node creation operator~\cite{Wood2012}.  Recollect that
nodes in $G$ have identity.  In order to create new nodes, a mechanism
to assign new identifiers is required.  One way to accomplish that is
using a Skolem function, a concept originally used in predicate logic.
Informally, node creation returns a graph with additional nodes
representing matches of an input pattern in the graph and identifiers
assigned by Skolem functions, and new edges connecting the new nodes
to elements of the matches, as specified by the pattern.  The web-site
management language StruQL outputs new nodes in a \insql{create}
clause, corresponding to the node creation operation with a Skolem
function to create the object
ids~\cite{Fernandez:1997:QLW:262762.262763}.  The GOOD language, based
on an object-oriented model, provides an \insql{abstraction} operator
that allows to create new nodes to represent multiple nodes based on
shared properties~\cite{Gyssens1994}.  

A temporal generalization of the node creation operator has not been
proposed for evolving graphs to our knowledge.  However, the G* system
supports SQL-style aggregation of the data using the
\insql{AggregateOperator} per graph snapshot, which allows a limited
version of summarization to be performed~\cite{Labouseur2015}.

Similarly, several variants of temporal aggregation of relational data
can be found in the literature.  The ``standard'' temporal aggregation
in \tra is a direct extension of the non-temporal aggregation operator
that does not change the temporal resolution of the data (see, for
instance, the discussion and example 10 in the work of Dign{\"{o}}s,
et al.~\cite{Dignos2012}).  However, Li et al. proposed a nice general
window aggregate for data streams~\cite{Li2005} that can be easily
applied to temporal relational data.  The semantics of the window
aggregates are based on a sliding window specification, consisting of
a range and slide of the window based on the desired data attribute,
as long as the attribute has a domain with a total order.  The range
specifies the width of the window, e.g., 100 seconds or 100 rows.  The
slide specifies how windows are formed in relation to each other,
whether overlapping or not.  Since valid time data has total order in
the time domain, overlapping and non-overlapping time windows can be
created with this approach.  We are not aware of any proposal for an
operator capable of changing the temporal resolution of evolving
graphs, besides our own.

