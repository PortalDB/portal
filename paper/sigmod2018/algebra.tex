\section{Temporal node creation}
\label{sec:semantics}

\subsection{Preliminaries}

We assume a linearly ordered, discrete time domain $\Omega^T$ where
time {\em instances} have limited precision.  In temporal relational
databases, a {\em valid-time} temporal relation schema is represented
as $R = (A_1, \ldots, A_m~|~T)$, where $A_1, \ldots, A_m$ are
nontemporal attributes with domain $\Omega_i$ and $T$ is a temporal
attribute over $\Omega^T \times \Omega^T$.  The timestamp attribute is
special and thus separated by a $|$ symbol in the list of attributes.
\eat{It is also always listed last.  }This is called {\em tuple
  timestamping}~\cite{Montanari2009}, since each tuple in a relation
is associated with a single time attribute during which it is known to
hold.  In principle, the time attribute can be a single time instant
or a set of instants.  We use periods to compactly represent the
constituent time points.  This is a common representation technique,
which does not add expressive power to the data model, compared to
associating each tuple with a single time
instant~\cite{DBLP:conf/ictl/Chomicki94}.  Following the SQL:2011
standard~\cite{DBLP:journals/sigmod/KulkarniM12}, a period (or
interval) $t = [s, e)$ represents a discrete contiguous set of time
  instances from domain $\Omega^T$, starting from and including the
  start time $s$, continuing to but excluding the end time $e$.

\subsection{TGraph Model}
We now describe the logical representation of an evolving graph,
called a \tg.  A \tg represents a single graph, and models the
evolution of its topology and of vertex and edge properties.  

\subsection{Attribute-based node creation}

\subsection{Window-based node creation}
