\begin{abstract}

Graphs are used to represent a plethora of phenomena, from the Web and
social networks, to biological pathways, to semantic knowledge
bases. Arguably the most interesting and important questions one can
ask about graphs have to do with their evolution. Which Web pages are
showing an increasing popularity trend? How does influence propagate
in social networks? How does knowledge evolve?

This paper proposes a logical model of an evolving graph which
captures evolution of graph topology, and of attributes of vertices
and edges, and a compositional algebra over that model.  It includes
such operations as temporal selection, subgraph, aggregation,
set-based operators, and a rich class of analytics.  \eat{We introduce
temporal primitives that allow to reduce temporal operators of our
algebra into nontemporal ones. 
We show that our algebra is
relationally complete and is sufficient to express a wide range of
common use cases.
} We formally study the properties of our algebra and show that it is 
sufficient to conveniently and concisely express a wide range of
common use cases.

\eat{In this paper we address the need to enable {\em systematic support}
for scalable querying and analytics over {\em evolving graphs}.  We
propose a representation of an evolving graph, called a \tg, which
captures evolution of graph topology, and of attributes of vertices
and edges, continuously through time.  We develop a compositional \tg
algebra that includes
}

We developed several in-memory representations for our model in scope
of Apache Spark, an open-source distributed data processing framework.
Experimental results on real datasets demonstrate that no single
representation is most efficient for all operations, but that switching
between representations in a complex query can lead to good
performance.

\eat{We present \ql, a system that
implements our model and algebra in scope of Apache Spark, an
open-source distributed data processing framework.  We develop
multiple physical representations of evolving graphs and study the
trade-offs between structural and temporal locality.  We provide an
extensive experimental evaluation on real datasets, demonstrating that
careful engineering can lead to good performance.}

\end{abstract}
