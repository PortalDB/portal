\subsection{Examples}
\label{sec:example}

With the \ql operators in place, we can now show how the motivating
queries can be specified.

Question: In a sparse communication network, on what time scale can we
detect communities?

Process: a) compute new nodes with temporal windows of a set size; b)
compute connected components analytic; c) compute new nodes grouping
nodes by their component, computing their size; d) filter out nodes
that represent communities too small to be useful (e.g., of 1-2
people).  Examine the output and repeat from step a) with a different
temporal window size.

Query: $\sigma_{size > 2,true}($ \\ $ _{component}\vartheta_{1
  change,exists,exists,count(name),any}($ \\ $ pregel_{cc} (_v\vartheta_{6
  months,exists,always,first(name),any}( \ttt))))$

Question: Are there high influence nodes and is that behavior
persistent in time?

Process: a) compute in-degree (prominence) of each node for each time
interval it changed; b) compute new nodes with temporal window equal
to the overall timespan, collecting degree data into a map; c) apply
map to compute the coefficient of variation from the interval-degree
map.  Examine the output.

Query: $map_{stdev(degree)/mean(degree)*100}($ \\ 
$ _v\vartheta_{lifetime,exists,exists,map(degree),any}(
\gamma_{true,true,dst,1,count}( \ttt)))$

Question: How has graph in-degree centrality changed over time?

Process: a) compute new nodes with temporal windows of a set size; b)
compute in-degree of each node; c) compute new node grouping all nodes
co-existing in time into one, computing the sum, count, and max of
degree; d) apply map to compute the degree centrality.

Query: $map_{(max*count-sum)/(count^2-3count+2)}($ \\
$ _1\vartheta_{1 change,exists,exists,max(degree)\&sum(degree)\&count(degree),any}($ \\
$ \gamma_{true,true,dst,1,count}( _v\vartheta_{2 months,exists,always,any,any}( \ttt))))$
