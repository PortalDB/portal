\section{Conclusions and Future Work}
\label{sec:conc}

In this paper we presented an interval-based model of evolving graphs,
and proposed a composable \tg algebra that supports a rich set of
operations.  It is in our immediate plans to develop a declarative
syntax for the algebra, making it accessible to a wider audience of
users.  We described an impementation of \ql in scope of Apache Spark,
and studied performance of operations on different physical
representations.  Interestingly, different physical implementations
perform best for different operations, opening up avenues for
rule-based and cost-based optimization.  Developing a query optimizer
for \ql is in our immediate plans.

% Support for interval semantics: change preservation

% Support for recursion

% Support for path queries - quesions of efficiency and of semantics
% --- preserving soundness of the model

% Systems work: evaluate alternative physical representations for
% temporal property graphs.

\eat{ While the Portal language is extensive, it is by no means
  complete. We recognize that there are a number of operations that
  are currently not supported but would be useful to potential users:}

\eat{\begin{enumerate}
\item Temporal pattern matching.  While aggregation provides an
  ability to detect some patters, a more general temporal-structural
  pattern mining is needed.  Work on specifying structural patterns in
  graphs is ongoing, however, at the time of this writing we are not
  aware of a general approach for specifying structural patterns that
  also have a time dimension.  For example, how should the user
  specify that he/she is looking for small strongly connected
  components that exhibit consistent growth over a period of time?
  Some work on this front has been done by Chan et
  al.~\cite{Chan2008,Kan2009}.
\item Structural select (subgraph). 
\item Other kinds of temporal select besides by interval, such as with
  predicates, similar to snapshot selection support
  in~\cite{Khurana2013}. 
\item Across-time analytics (unlike snapshot-based analytics,
  e.g. pagerank) like the centrality metric for dynamic networks,
  where influence of a vertex propagates through time.
\item Anything that would return not another tgraph.  This could be
  measures of a whole graph (measure of centrality, degree
  distribution, diameter, etc.) or searching that returns a set of tgraphs,
  e.g. frequent pattern mining).
\end{enumerate}}
