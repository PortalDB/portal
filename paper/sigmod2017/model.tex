\section{Data Model}
\label{sec:model}

%\subsection{Preliminaries}
%\label{sec:model:prelim}

Following the SQL:2011
standard~\cite{DBLP:journals/sigmod/KulkarniM12}, a period (or
interval) $\bp = [s, e)$ represents a discrete set of time instances,
  starting from and including the start time $s$, continuing to but
  excluding the end time $e$.  Time instances contained within the
  period have limited precision, and the time domain has total order.

\eat{\begin{definition}[Time period]
A {\em time period} \\$\bp = [s, e)$ is an interval of the time
  domain, subject to the constraint $s < e$.
\label{def:period} 
\end{definition}}

We quantify relationships between time periods using the following
Allen's relations~\cite{allen83} with equality: \predName{meets},
\predName{overlaps}, \predName{starts}, \predName{finishes},
\predName{during}, and \predName{equals}.  We will use
$\pred{\bp}{contains}{\bq}$ as a shorthand for
$\pred{\bp}{starts}{\bq} \vee \pred{\bp}{finishes}{\bq} \vee
\pred{\bp}{during}{\bq} \vee \pred{\bp}{equal}{\bq}$.  We denote by
$\bp \cap \bq$ the intersection of $\bp$ and $\bq$.

\begin{figure}[t!]
\centering
%\includegraphics[width=3.5in]{figs/T1_rel_tab.pdf}
\includegraphics[width=3in]{figs/T1_rel.pdf}
\caption{\tg \insql{T1}.}
\label{fig:tg_ve}
\end{figure}

\begin{figure}[t!]
%\centering
\includegraphics[width=3.4in]{figs/T1_graphs.pdf}
\vspace{-0.5cm}
\caption{Representative graphs of \tg \insql{T1}.}
\vspace{-0.4cm}
\label{fig:tg_rg}
\end{figure}


\eat{It will be useful to quantify relationships between time periods $p$
and $q$ using the following Allen's relations~\cite{allen83}.}

\eat{\begin{itemize}
\item $p equal q$, defined as $p.start = q.start \wedge p.end = q.end$
\item $p overlaps q$, defined as $p.end > q.start$
\item $p meets q$, defined as $p.end = q.start$
\item $p during q$, defines as $p.start > q.start \wedge p.end < q.end$ 
\item $p starts q$, defined as $p.start = q.start \wedge p.end < q.end$
\item $p finishes q$, defined as $p.start > q.start \wedge p.end = q.end$
\end{itemize}}

%\subsection{Representative graphs of a \tg}
\label{sec:model:rg}

The basic building block of our model is a \tg, which represents a
single graph that evolves continuously over time, and is defined over
a set of vertices $V$, edges $E\subseteq V \times V$, vertex
attributes \avv, and edge attributes \aee.

\begin{definition}[TGraph]
A \tg $\tgg(g, p)$ is a valid-time period-based relation that
associates a state of the graph $g$ with a time period $p$ during
which the graph is in that state.


Each $g = (V_g, E_g, \avv_g, \aee_g)$, with $V_g \subseteq V$, $E_g
\subseteq E$, $\avv_g \subseteq \avv$, $\aee_g \subseteq \aee$.

\tgg is temporally coalesced:
\vspace{-0.2cm}
\begin{multline}
\forall \tgg(g,p)~\nexists \tgg(g,p')~~| \\
        \pred{p}{meets}{p'}~\lor~\pred{p}{contains}{p'}~\lor~\pred{p}{overlaps}{p'}
\end{multline}

\tgg represents evolution of a single graph over time:
\begin{multline}
\forall \tgg(\_,p)~\nexists \tgg(\_,p')|~\pred{p}{contains}{p'}~\lor~\pred{p}{overlaps}{p'}
\end{multline}

We refer to each $g$ that appears in some tuple $(g,p)$ of \tgg as a
representative graph of \tgg during period $p$.
\label{def:tg_abstract}
\end{definition}

An example of a \tg is given in Figure~\ref{fig:tg_rg}, where 5
representative graphs (states) of \insql{T1} are associated with 5
consecutive periods $p_1=[1/15, 2/15), \ldots, p_5=[7/15, 10/15)$.} It
    is not required that time periods of a \tg be consecutive and with
    no gaps.

    Vertices and edges of a \tg are not required to be homogeneous in
    terms of their schemas.  In line with several popular graph
    databases, we use the property graph model~\cite{GraphDB} to
    represent vertex and edge attributes.  Each vertex and edge of $g$
    during period $p$ is associated with a (possibly empty) {\em set}
    of properties, and each property is represented by a key-value
    pair.  Values are not restricted to be of atomic types, and may,
    e.g., be maps or tuples.

As is typical in interval-based models, each representative graph is
assumed to exists continuously, and remains unchanged, during the
associated period.  By this we mean that graph topology (identities of
its vertices and edges), and properties of each vertex and edge remain
unchanged.

The statement that \tgg is temporally coalesced means that each fact
(unchanging state of the graph) is represented exactly once for each
time period of maximal length when it
holds~\cite{DBLP:conf/vldb/BohlenSS96}.\eat{ This is the standard
  meaning of the term ``coalesce'' in temporal databases, and is
  unrelated to the SQL coalesce function.}  For interval-based models,
requiring that relations be coalesced is both space-efficient and
avoids semantic ambiguity (see~\cite{DBLP:reference/db/JensenS09k}
Fig. 2 and its description).






We use valid-time temporal SQL
relations~\cite{DBLP:conf/vldb/BohlenSS96} to model evolution of graph
topology and of vertex and edge attributes.  We term this logical
representation a \tg.  An example of a \tg that shows evolution of a
co-authorship network is given in Figure~\ref{fig:tg_ve}.  Note that
\tg represents evolution of a {\em single graph} over time.
%
We use the property graph model~\cite{GraphDB} to represent vertex and
edge attributes: each vertex and edge during period $\bp$ is associated
with a (possibly empty) {\em set} of properties, and each property is
represented by a key-value pair.  Property values are not restricted
to be of atomic types, and may, e.g., be maps or tuples.

\tg uses the point-stamped temporal
model~\cite{DBLP:reference/db/Toman09}, associating a fact (existence
of a vertex or edge, or an assignment of a value to a vertex or edge
attribute) with a time instant.  We use intervals to compactly
represent their constituent time instants.  This is a common
representation technique, which does not add expressive power to the
data model~\cite{DBLP:conf/ictl/Chomicki94}.  We now give a formal
definition of a \tg.

\begin{definition}[TGraph]
A \tg is a pair $\tve=(\tv, \te)$. \tv is a valid-time temporal SQL
relation with schema $\tv(\underline{v}, \underline{\bp})$ that
associates a vertex with the time period during which it is
present. \te is a valid-time temporal SQL relation with schema
$\te(\underline{v_1}, \underline{v_2}, \underline{\bp})$, connecting
pairs of vertices from \tv.  \tv and \te are temporally coalesced:
\vspace{-0.3cm}
\begin{multline}
\vspace{-0.3cm}
\forall \tv(v, \bp)~~\nexists \tv(v, \bq)~~| \\
                       \pred{\bp}{meets}{\bq'}~\lor~\pred{\bp}{contains}{\bq}~\lor~\pred{\bp}{overlaps}{\bq}
\label{def:tg:c2}
\end{multline}
\vspace{-0.7cm}
\begin{multline}
\forall \te(v_1, v_2, \bp)~~\nexists \te(v_1,v_2, \bq)~~| \\
                       \pred{\bp}{meets}{\bq}~\lor~\pred{\bp}{contains}{\bq}~\lor~\pred{\bp}{overlaps}{\bq}
\label{def:tg:c3}
\end{multline}

An edge connects a pair of vertices that exist at the time when the edge exists:
\vspace{-0.3cm}
\begin{multline}
\forall \te(v_1, v_2, \bp)~~\exists \tv(v_1, \bp_1), \tv(v_2, \bp_2)~~| \\
                       \pred{\bp_1}{contains}{\bp}~\wedge~\pred{\bp_2}{contains}{\bp}
\label{def:tg:c1}
\end{multline}
\vspace{-0.5cm}

\tve optionally includes vertex and edge attribute relations \tav
and \tae.  Both are coalesced, and associate sets of properties with
existing vertices and edges.
\vspace{-0.3cm}
\begin{multline}
\forall \tav(v, p, a)~~\nexists \tav(v, p', a)~~| \\
                       \pred{p}{meets}{p'}~\lor~\pred{p}{contains}{p'}~\lor~\pred{p}{overlaps}{p'}
\label{def:tg:c4}
\end{multline}
\vspace{-0.5cm}
\begin{multline}
\forall \tav(v, p, a)~~\exists \tv(v,p')~~|~~\pred{p'}{contains}{p}
\label{def:tg:c5}
\end{multline}
\vspace{-0.5cm}
\begin{multline}
\forall \tae(v_1, v_2, p, a)~~\nexists \tae(v_1, v_2, p', a)~~| \\
                       \pred{p}{meets}{p'}~\lor~\pred{p}{contains}{p'}~\lor~\pred{p}{overlaps}{p'}
\label{def:tg:c6}
\end{multline}
\vspace{-0.5cm}
\begin{multline}
\forall \tae(v_1, v_2, p, a)~~\exists \te(v_1,v_2,p')~~|~~\pred{p'}{contains}{p}
\label{def:tg:c7}
\end{multline}
\vspace{-0.6cm}
\label{def:tg}
\end{definition}
\vspace{-0.3cm}

\eat{Vertices and edges of a \tg are not required to be homogeneous in
terms of their schemas.  In line with several popular graph databases,
we use the property graph model~\cite{GraphDB} to represent vertex and
edge attributes.  Each vertex and edge during period $\bp$ is
associated with a (possibly empty) {\em set} of properties, and each
property is represented by a key-value pair.  Values are not
restricted to be of atomic types, and may, e.g., be maps or tuples.}

Graphs may be directed or undirected.  For undirected graphs we choose
a canonical representation of an edge, with $v_1 \leq v_2$ (self-loops
are allowed).  Because we use the source and destination vertex id
pair as the identifier for the edges, at most two edges can exist
between any two vertices (one in each direction) at any time point.
That is, we do not support multigraphs.

\eat{Conditions~\ref{def:tg:c2} and~\ref{def:tg:c3} in
Definition~\ref{def:tg} state that \tv and \te are
coalesced~\cite{DBLP:conf/vldb/BohlenSS96}, i.e., that each vertex and
edge is represented exactly once for each time period of maximal
length when it is present.  Conditions~\ref{def:tg:c4}
and~\ref{def:tg:c6} state that the attribute relations are coalesced,
i.e., that an attribute is represented exactly once for each time
period of maximal length in which its value did not change.}

Conditions~\ref{def:tg:c2},~\ref{def:tg:c3},~\ref{def:tg:c4},
and~\ref{def:tg:c6} in Definition~\ref{def:tg} state that all
relations of \tve are coalesced~\cite{DBLP:conf/vldb/BohlenSS96} ---
each fact (existence of a vertex or edge, or an assignment of a value
to a vertex or edge attribute) is represented exactly once for each
time period of maximal length during which it holds.  Consider
Figure~\ref{fig:tg_ve} and note that there is a single tuple for $v_2$
in \tv, but two tuples for $v_2$ in \tav, because $school=CMU$ was
added at time 5/15.
%
\eat{The statement that \tg is temporally coalesced means that each
  fact (unchanging state of a vertex or edge, or unchanging value of a
  vertex or edge attribute) is represented exactly once for each time
  period of maximal length when it
  holds~\cite{DBLP:conf/vldb/BohlenSS96}.}\eat{ This is the standard
  meaning of the term ``coalesce'' in temporal databases, and is
  unrelated to the SQL coalesce function.}  For point-stamped temporal
models, requiring that relations be coalesced avoids semantic
ambiguity (see~\cite{DBLP:reference/db/JensenS09k} Fig. 2 and its
description).

\eat{It is not required that a vertex or an edge be represented in
  \tav and \tae --- a vertex or edge that never had any associated
  properties will have no corresponding tuples in \tav or \tae.}

\eat{ Definitions~\ref{def:tg_abstract} and~\ref{def:tg} give two
  alternative views of graph evolution.  In
  Definition~\ref{def:tg_abstract} the graph is represented in its
  entirety, with a change of state occurring whenever there is a
  change in graph topology or in its vertex or edge properties.  In
  other words, time periods here are at their finest granularity.  In
  contrast, Definition~\ref{def:tg} decouples evolution of graph
  vertices, edges and attribute values, coalescing each relation
  independently of the others.  This makes graph maintenance and
  manipulation more manageable.  }

\eat{The \tv and \te relations of Definition~\ref{def:tg}, although only
partially enforcing the constraints, can be implemented as follows (as
per SQL:2011):}

\eat{ Condition~\ref{def:tg:c1} in Definition~\ref{def:tg} states the
  natural integrity constraint that an edge linking two vertices can
  only exist during a time when both vertices exist. Here,
  $\pred{p}{contains}{q}$ is the Allen \predName{contains} relation
  with equality~\cite{allen83}, defined as $\bp.start \leq \bq.start
  \wedge \bp.end \geq \bq.end$.  Consider the vertex-edge representation
  of \tg \insql{T1} in Figure~\ref{fig:tg_ve}, where edge $e(v_1,v_2)$
  exists during the entire period when both $v_1$ and $v_2$ exist,
  while edge $e(v_2,v_3)$ exists only for a portion of the period when
  both $v_2$ and $v_3$ exist.}

\eat{
\begin{scriptsize}
\begin{verbatim}
CREATE TABLE TV (
  v LONG,
  pstart DATE,
  pend DATE,
  PERIOD for p (pstart, pend),
  PRIMARY KEY (v, p WITHOUT OVERLAPS) )
CREATE TABLE TE (
  v1 LONG,
  v2 LONG,
  pstart DATE,
  pend DATE,
  PERIOD for p (pstart, pend),
  PRIMARY KEY (v1, v2, p WITHOUT OVERLAPS),
  FOREIGN KEY (v1, PERIOD p) REFERENCES TV(v, PERIOD p),
  FOREIGN KEY (v2, PERIOD p) REFERENCES TV(v, PERIOD p) )
\end{verbatim}
\vspace{-0.3cm}
\end{scriptsize}
}\eat{ The \tv and \te relations of Definition~\ref{def:tg}, although only
partially enforcing the constraints, can be implemented per SQL:2011
standard. Unfortunately, an implementation will not guarantee that \tv
and \te are coalesced.  It is possible to ensure that time periods
corresponding to a vertex (resp. edge) do not overlap or contain each
other, but it is not possible to prevent the existence of two adjacent
time periods for a vertex or edge, i.e., periods $\bp$ and $\bq$ for which
$\pred{\bp}{meets}{\bq}$.  That is, there may well be two tuples in \tv
for $v_1$ in Figure~\ref{fig:tg_ve}, one associated with $[1/15,
  2/15)$ and the other with $[2/15, 7/15)$.}

\eat{The relations of \tve can be implemented in temporal SQL, as per the
SQL:2011 standard~\cite{DBLP:journals/sigmod/KulkarniM12}, but this
implementation will not ensure that relations are coalesced.  SQL:2011
does not provide a way to specify that a relation be coalesced, and
does not enable efficient mechanisms to implement the coalesce
operator.  Coalescing requires that the system automatically merge
adjacent and overlapping time periods.  This operation, which is
similar to duplicate elimination in conventional databases, has been
extensively studied in the
literature~\cite{DBLP:conf/vldb/BohlenSS96,DBLP:journals/sigmod/Zimanyi06},
but is not supported by the standard.}\eat{ Note that, while it is an
  important property of our logical model that time periods be
  coalesced, eagerly coalescing is both expensive and, in some cases,
  unnecessary.  We will discuss lazy coalescing in
  Section~\ref{sec:sys:coal}.}

In the \tg representation of Definition~\ref{def:tg}, vertex and edges
attributes are stored as collections of properties.  That said,
Definition~\ref{def:tg} presents a logical data structure that admits
different physical representations, including, e.g., by a columnar
representation of properties (each property in a separate relation,
supporting different change rates), by a hash-based representation
of~\cite{DBLP:conf/sigmod/SunFSKHX15}, or in some other way. We leave
an experimental comparison of different physical representations of
vertex and egde attributes to future work.

\eat{
Definition~\ref{def:tg} gives a columnar representation of vertex and
edge attributes.  Such a representation naturally supports {\em schema
  evolution} and can lead to more efficient physical representations,
especially when values of different attributes change at different
rates.  That said, our support of non-atomic types allows to combine
multiple, or even all, vertex (resp. edge) attributes in a single
attribute relation, as is the case in \tav Figure~\ref{fig:tg_ve}.}

Our choice to use attribute relations is in contrast to representing
vertex and edge attributes as part of \tv and \te.  The main reason is
to streamline the enforcement of referential integrity constraints of
Definition~\ref{def:tg}.  Consider again the example in
Figure~\ref{fig:tg_ve}.  If vertex attributes were stored as part of
\tv, then there would be two tuples for $v_2$ in this relation whose
validity periods overlap with that of edge $e(v_1, v_2)$ --- one for
each $[2/15, 5/15)$ and $[5/15, 7/15)$.  This would in turn require
    that $e(v_1, v_2)$ be mapped to two tuples in \tv as part of
    referential integrity checking on $v_2$.  Matching a tuple with a
    set of tuples in the referenced table, while supported by the
    SQL:2011 standard, is potentially inefficient, and we avoid it in
    our representation.
%
\eat{ We will discuss how integrity of the model is enforced, and how
  this is implemented, in Sections~\ref{sec:algebra}
  and~\ref{sec:sys}.}Another reason for optional \tav and \tae is
that in many cases we are interested in applying operations (e.g.,
analytics) only to graph topology, and in that case \tv and \te are
sufficient.
%
\eat{ Non-key attributes of $V$ and $E$ are not restricted to be of
  atomic types, but may, e.g., be maps or tuples.  Because of this,
  \ql supports {\em schema evolution} in a \tg in much the same way as
  is done in popular (non-temporal) graph databases like Neo4j.  At an
  extreme, the vertex (resp. edge) relation will have a single
  unstructured non-key attribute, which would store all attribute
  information.  This representation has the usual advantages and
  disadvantages --- flexibility of a schema-less representation at the
  expense of missed performance optimization opportunities afforded by
  a structured representation.}

A snapshot of a \tg, denoted $\tau^s_c (\tve)$ (``s'' stands for
``snapshot''), is the state of \tve at time point $c$. It can be
obtained by taking a timeslice at $c$ from all constituent relations
of \tve.  It is sometimes convenient to use intervals to compactly
represent consecutive value-equivalent snapshots of \tve ---
timeslices in which no change occurred in graph topology, or in vertex
and edge attributes.  We use the term {\em representative graph} to
refer to such snapshots, since they represent an interval.
Figure~\ref{fig:tg_rg} shows \insql{T1} as a sequence of
representative graphs.

\eat{\begin{definition}[\rgs]
A {\em representative graphs} view of a \tg is a sequence of
(non-temporal) graphs associated with a sequence of consecutive
non-overlapping time periods.
\label{def:rgs} 
\end{definition}}

\eat{This view, to which we refer as \rgs of \insql{T1}, is computed from
properly coalesced $V$, $E$, $A^{V}$, and $A^{E}$ by (1) enumerating
all time points during which some change occurred, i.e., that appear as
either the start or the end of some time period, and (2) generating
pairs of adjacent time points. See Appendix for details.  \rgs is an
important abstraction that will be useful to us in the next section,
where we present \tg algebra.
}

%\subsection{Switching between representations}  
%\label{sec:model:switch}

\eat{
\trg and \tve are equivalent in terms of the information they contain.
\trg can be computed from \tve by (1) enumerating all time points
during which some change occurred, i.e., those that appear as either
the start or the end of some time period in \tv, \te, \tav, and \tae;
(2) computing \trg periods by generating pairs of adjacent time points
from step 1, and sorting them by $start$; (3) constructing a graph $g$
for each time period $\bp$ by selecting tuples from \tv, \te, \tav, and
\tae that overlap with $\bp$.
}

\eat{
To compute \tve from \trg, we iterate over $\trg (g,p)$, computing
$\tv_g(v,p)$ (similarly for $\te_g$, $\tav_g$, $\tav_g$) for each
tuple, collecting $\tv_g \rightarrow \tv$, $\te_g \rightarrow \te$,
$\tav_g \rightarrow \tav$, $\tae_g \rightarrow \tae$, and coalescing
each \tv, \te, \tav, \tae.
}

\eat{Next, we will present operators of \tg algebra, stating their
semantics over both\trg and \tve.}  \eat{We will then discuss how each
  operator is implemented over several possible physical
  representations of these logical data structures
  (Section~\ref{sec:sys}).}

