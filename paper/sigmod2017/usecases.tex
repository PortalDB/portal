\subsection{Use cases and algebra by example}
\label{sec:cases}

An interaction network is one typical kind of an evolving graph.  It
repesents people as vertices, and interactions between them such as
messages, conversations and endorsements, as edges.  Information
describing people and their interactions is represented by vertex and
edge attributes.  One easily accessible interaction network is the
wiki-talk dataset (\url{http://dx.doi.org/10.5281/zenodo.49561}),
containing messaging events among Wikipedia contributors over a
13-year period.  Information available about the users includes their
username, group membership, and the number of Wikipedia edits they
made.  Messaging events occur when users post on each other's talk
pages.

We now present common analysis tasks that motivate the operators of
our algebra, \tga.  We are primarily interested in analyses of the
evolution of the phenomena the graph represents, while also supporting
simpler point queries.

{\bf Example 1.}  In an interaction graph, vertex centrality is a
measure of how important or influential people are.  Over a dozen
different centrality measures exist, providing indicators of how much
information ``flows'' through the vertex or how the vertex contributes
to the overall cohesiveness of the network.  Vertex importance
fluctuates over time.  To see whether the wiki-talk graph has
high-importance vertices, and how stable vertex importance is over
time during a particular period of interest, we can look at a subset
of the graph that corresponds to the period of interest, compute an
importance measure, such as in-degree, for each vertex and for each
point in time, and finally calculate the coefficient of variation per
vertex.

\eat{This example demonstrates a need to select a subset of the data
corresponding to the period of interest, compute in-/out-degree for
each vertex at each point in time, and compute a single measure across
time for each vertex.  Computation of degree is a simple example of a
non-temporal {\em aggregation} operation as defined by the taxonomy of
Wood~\cite{Wood2012}.  Aggregation computes a value for each vertex
based on its neighbors and can be used for a wide variety of analyses.
SocialScope~\cite{Amer-Yahia2009} is one of the languages that proposes
an aggregation operation and demonstrates its many uses.  We introduce
a temporal version of aggregation.}

{\bf Question:} What are the high-influence nodes over the past 5
years, and is their influence persistent over time?

\begin{enumerate}[noitemsep,itemindent=\dimexpr\labelwidth+\labelsep\relax,leftmargin=0pt]
\item Select a subset of the data representing the 5 years of
  interest, using a common temporal operator slice($\tau$):

\begin{center}
\vspace{-0.2cm}
$\ttt_1 = \slice{[2010,2015)}{wikitalk}$
\vspace{-0.2cm}
\end{center}

\item Compute in-degree (prominence) of each vertex during each time
  point.  This is an example of the {\em aggregation} operation, a
  common operation on non-temporal graphs, as defined by the taxonomy
  of Wood~\cite{Wood2012}.  Aggregation computes a value for each
  vertex based on its neighbors.  SocialScope~\cite{Amer-Yahia2009} is
  one of the languages that proposes an aggregation operation and
  demonstrates its many uses.  We introduce a temporal version of
  aggregation (listed here with default arguments omitted for
  readability):

\begin{center}
\vspace{-0.2cm}
%$\ttt_2 = \agg{msg=(dst,p,1),red=count}{\ttt_1}$
$\ttt_2 = \insql{agg}^T(\mathsf{f_m=1},\mathsf{f_a=count},\mathsf{pname=degree},\ttt_1)$
\vspace{-0.2cm}
\end{center}

\item Aggregate degree information per vertex across the timespan
  of $\ttt_2$, collecting values into a map.  This is an example of
  aggregation based on temporal window, which we implement with the
  temporal node creation operator:

\begin{center}
\vspace{-0.2cm}
$\ttt_3 =\insql{node}^T_w(\mathsf{w=lifetime},\mathsf{f_v=map(degree)},\ttt_2)$
%$\ttt_3 = \ncr{v}{W=lifetime,red=map(degree)}{\ttt_2}$
\vspace{-0.2cm}
\end{center}

\item Transform the attributes of each vertex to compute the
  coefficient of variation from the map of degree values, using the
  temporal vertex-map operator:

\begin{center}
\vspace{-0.2cm}
%$\ttt_4 = \map_{\mathsf{stdev(degree)/mean(degree)*100}}(\ttt_3)$
$\ttt_4 = \vmap{\mathsf{f_v=stdev(degree)/mean(degree)*100}, \ttt_3}$
\vspace{-0.2cm}
\end{center}

\end{enumerate}

{\bf Example 2.}  Graph centrality is a popular measure to evaluate
how connected/centralized the community is.  Low centrality may
indicate that the community is disjointed or communicates poorly.  An
interesting measure by itself, it is subject to change as
communication patterns evolve or high influencers appear or disappear.
In sparse interaction graphs there is an additional question of what
time scale to consider: if two people communicated on May 16, 2010,
how long do we consider them connected?

\eat{This example demonstrates a need to compute graph centrality at every
point in its lifetime and to do so at different temporal resolution.
For most graph centrality measures, the two-step process involves
first calculating some measure, such as in-degree, for each graph
vertex, and then accumulating them into one.  This calls for an
operation that can group multiple vertices, in this case all of them,
into a new vertex.  Wood terms this operation {\em node creation} and
shows that many graph languages such as GraphQL~\cite{He2008} support
it if there is a need to output new nodes that were not part of the
input.  Creating a single vertex to represent the whole graph is one
way to support computation of some whole-graph measure, but, as we
show below, node creation is useful for other types of analyses.  Use
of temporal windows is also important here to consider different
temporal resolution, which is similar to temporal aggregation in
temporal relational databases.  Our temporal node creation operator
can create new nodes from structure or temporal information or both.}

{\bf Question:} How has graph in-degree centrality changed over time?

\begin{enumerate}[noitemsep,,itemindent=\dimexpr\labelwidth+\labelsep\relax,leftmargin=0pt]
\item Compute new vertices with temporal windows of a set size.  Use
  of temporal windows is used here to consider different temporal
  resolution, which is similar to temporal aggregation in temporal
  relational databases.  Our temporal node creation operator can
  create new nodes from structure or temporal information or both.  In
  the example above, we created new nodes to represent all of each
  node's history.  Here we use smaller temporal window, e.g. 2 months,
  and can vary that to investigate the effect of temporal resolution:

\begin{center}
\vspace{-0.2cm}
$\ttt_1 = \ncr{v}{W=2 months,Q_E=always}{wikitalk}$
\vspace{-0.2cm}
\end{center}

\item Compute in-degree of each vertex:

\begin{center}
\vspace{-0.2cm}
$\ttt_2 = \agg{msg=(dst,p,1),red=count}{\ttt_1}$
\vspace{-0.2cm}
\end{center}

\item Compute new vertex grouping all vertices co-existing in time
  into one, computing the sum, count, and max of degree.  For most
  graph centrality measures, the two-step process involves first
  calculating some measure, such as in-degree, for each graph vertex,
  and then accumulating them into one.  This calls for an operation
  that can group multiple vertices, in this case all of them, into a
  new vertex.  Wood terms this operation {\em node creation} and shows
  that many graph languages such as GraphQL~\cite{He2008} support it
  if there is a need to output new nodes that were not part of the
  input.  Creating a single vertex to represent the whole graph is one
  way to support computation of some whole-graph measure, but, as we
  show throughout, node creation is useful for other types of
  analyses.  

\begin{center}
\vspace{-0.2cm}
$\ttt_3 = \ncr{1}{red=max(deg)\&sum(deg)\&count(deg)}{\ttt_2}$
\vspace{-0.2cm}
\end{center}

\item Apply map to compute the degree centrality:

\begin{center}
\vspace{-0.2cm}
$\ttt_4 = \map_{\mathsf{(max*count-sum)/(count^2-3*count+2)}}(\ttt_3)$
\vspace{-0.2cm}
\end{center}

\end{enumerate}

{\bf Example 3.}  Interaction networks are sparse because the edges
are so short-lived.  To see whether communities form and at what time
scales, we can vary the time scale and compute communities,
e.g. through connected components detection, group the vertices by the
community they form and calculate their size.  We can filter out
vertices that represent communities below a reasonable threshold, for
example of size smaller than two.

\eat{This example demonstrates a need to compute graph-wide analytics
  such as connected components for each point in time, create new
  vertices that represent some aspect of data of existing vertices,
  and compute subgraphs.  Graph-wide analytics on evolving graphs have
  been proposed previously in ImmortalGraph~\cite{Miao2015} and
  G*~\cite{Labouseur2015}, including PageRank, weakly connected
  components, and source-source shortest path.}

{\bf Question:} In a sparse communication network, on what time scale can we
detect communities?

\begin{enumerate}[noitemsep,itemindent=\dimexpr\labelwidth+\labelsep\relax,leftmargin=0pt]
\item Compute new vertices with temporal windows of a set size:

\begin{center}
\vspace{-0.2cm}
$\ttt_1 = \ncr{v}{W=6 months,Q_E=always,red=first(name)}{wikitalk}$
\vspace{-0.2cm}
\end{center}

\item Compute connected components.  We need to compute graph-wide
  analytics such as connected components for each point in time.
  Graph-wide analytics on evolving graphs have been proposed
  previously in ImmortalGraph~\cite{Miao2015} and
  G*~\cite{Labouseur2015}, including PageRank, weakly connected
  components, and source-source shortest path.

\begin{center}
\vspace{-0.2cm}
$\ttt_2 = pregel_{cc} (\ttt_1)$
\vspace{-0.2cm}
\end{center}

\item Compute new vertices grouping vertices by their component,
  accumulating total group size:

\begin{center}
\vspace{-0.2cm}
$\ttt_3 = \ncr{component}{W=1 change,red=count(name)}{\ttt_2}$
\vspace{-0.2cm}
\end{center}

\item Filter out vertices that represent communities too small to be
  useful (e.g., of 1-2 people).  This is an example of vertex subgraph:

\begin{center}
\vspace{-0.2cm}
$\ttt_4 = \subv{v.a.size > 2}{\ttt_3}$
\vspace{-0.2cm}
\end{center}

\end{enumerate}

Besides the examples above, graph queries commonly include retrieving
a specific node (which can be accomplished through a subgraph) and
k-hop neighborhood of a node.  While general transitive closure
requires recursion, which we do not support, k-hop neighborhoods can
be computed using composition operations like the one defined in
SocialScope~\cite{Amer-Yahia2009}.  We provide temporal versions of
subgraph and generalized edge creation.  Additionally, if multiple
sources of the same graph are available, it is useful to combine or
compare them, which dictates the need for temporal set-theoretic
operators.

In Section~\ref{sec:algebra} we formally define the operators of our
graph algebra.  In Section~\ref{sec:exp} we return to these three use
cases.  \eat{and show how our use cases can be expressed using these
  operators.}
