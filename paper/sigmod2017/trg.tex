\subsection{Representative graphs of a \tg}
\label{sec:model:rg}

The basic building block of our model is a \tg, which represents a
single graph that evolves continuously over time, and is defined over
a set of vertices $V$, edges $E\subseteq V \times V$, vertex
attributes \avv, and edge attributes \aee.

\begin{definition}[TGraph]
A \tg $\tgg(g, p)$ is a valid-time period-based relation that
associates a state of the graph $g$ with a time period $p$ during
which the graph is in that state.


Each $g = (V_g, E_g, \avv_g, \aee_g)$, with $V_g \subseteq V$, $E_g
\subseteq E$, $\avv_g \subseteq \avv$, $\aee_g \subseteq \aee$.

\tgg is temporally coalesced:
\vspace{-0.2cm}
\begin{multline}
\forall \tgg(g,p)~\nexists \tgg(g,p')~~| \\
        \pred{p}{meets}{p'}~\lor~\pred{p}{contains}{p'}~\lor~\pred{p}{overlaps}{p'}
\end{multline}

\tgg represents evolution of a single graph over time:
\begin{multline}
\forall \tgg(\_,p)~\nexists \tgg(\_,p')|~\pred{p}{contains}{p'}~\lor~\pred{p}{overlaps}{p'}
\end{multline}

We refer to each $g$ that appears in some tuple $(g,p)$ of \tgg as a
representative graph of \tgg during period $p$.
\label{def:tg_abstract}
\end{definition}

An example of a \tg is given in Figure~\ref{fig:tg_rg}, where 5
representative graphs (states) of \insql{T1} are associated with 5
consecutive periods $p_1=[1/15, 2/15), \ldots, p_5=[7/15, 10/15)$.} It
    is not required that time periods of a \tg be consecutive and with
    no gaps.

    Vertices and edges of a \tg are not required to be homogeneous in
    terms of their schemas.  In line with several popular graph
    databases, we use the property graph model~\cite{GraphDB} to
    represent vertex and edge attributes.  Each vertex and edge of $g$
    during period $p$ is associated with a (possibly empty) {\em set}
    of properties, and each property is represented by a key-value
    pair.  Values are not restricted to be of atomic types, and may,
    e.g., be maps or tuples.

As is typical in interval-based models, each representative graph is
assumed to exists continuously, and remains unchanged, during the
associated period.  By this we mean that graph topology (identities of
its vertices and edges), and properties of each vertex and edge remain
unchanged.

The statement that \tgg is temporally coalesced means that each fact
(unchanging state of the graph) is represented exactly once for each
time period of maximal length when it
holds~\cite{DBLP:conf/vldb/BohlenSS96}.\eat{ This is the standard
  meaning of the term ``coalesce'' in temporal databases, and is
  unrelated to the SQL coalesce function.}  For interval-based models,
requiring that relations be coalesced is both space-efficient and
avoids semantic ambiguity (see~\cite{DBLP:reference/db/JensenS09k}
Fig. 2 and its description).




