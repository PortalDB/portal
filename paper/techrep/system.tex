\section{System}
\label{sec:sys}

\begin{figure}[t!]
\begin{center}
\includegraphics[height=1.4in]{figs/architecture.pdf}
\caption{\ql system architecture.}
\label{fig:arch}
\end{center}
\end{figure}

Our \ql system implementation builds on GraphX, an Apache Spark
library, as depicted in Figure~\ref{fig:arch}.  Green boxes indicate
built-in components, while blue are those we added for \ql.  We
selected Apache Spark because it is a popular open-source system, and
because of its in-memory processing approach.  All language operators
on \tgs are available through the public API of the \ql library, and
may be used like any other library in an Apache Spark application.

\begin{figure}[t!]
\includegraphics[width=3.2in]{figs/sgp.pdf}
\caption{SG representation of T1 from Figure~\ref{fig:tg} (partial).}
\label{fig:sgp}
\end{figure}

\ql query execution follows the traditional query processing steps:
parsing, logical plan generation and verification, and physical plan
generation.  \ql re-uses and extends SparkSQL abstractions for these
steps.  A \ql query is rewritten into a sequence of operators, and
some operators are reordered to improve performance
(Section~\ref{sec:sys:optimization}).  We developed several different
physical representations (Section~\ref{sec:sys:datastructs}) and
partitioning strategies (Section~\ref{sec:sys:partition}) that are
selected at the physical plan generation stage.  The \tgs are read
from the distributed file system HDFS and processed by Spark Workers,
with the tasks assigned and managed by the runtime.  The System
Catalog contains information about each data set, including the
temporal and structural schema, and temporal bounds.

\begin{figure*}[th!]
\begin{minipage}{2.3in}
\centering
\includegraphics[width=2.2in]{figs/mg.pdf}
\caption{MG for T1 (partial).}
\label{fig:mg}
\end{minipage}
\begin{minipage}{2.3in}
\centering
\includegraphics[width=2.2in]{figs/mgc.pdf}
\caption{MGC for T1 (partial).}
\label{fig:mgc}
\end{minipage}
\begin{minipage}{2.3in}
\centering
\includegraphics[width=2.2in]{figs/og.pdf}
\caption{OG for T1 (partial).}
\label{fig:og}
\end{minipage}
\end{figure*}

\subsection{Query Optimization}
\label{sec:sys:optimization}

The declarative nature of \ql means that, like in SQL, queries can be
optimized for better performance.  \ql optimizations can be done by
operator reordering, application of best data structure and partition
strategy based on rules, and cost-based optimizations.  We now discuss
the first two options.

{\bf Operator reording.} Recollect that the logical order of
evaluation of a \ql query is selection, join, aggregation, and,
finally, analytics and projection.  We can show that there are classes
of queries which can be reordered without the loss of correctness.

{\bf Aggregation before join.}  Temporal aggregation produces a
smaller number of snapshots than its input.  Since the performance of
joins depends on the number of snapshots being structurally joined,
smaller number of snapshots will lead to faster performance.  Consider
this query:

\begin{small}
\begin{verbatim}
    TSelect Any V; Any E
    From    T1 TOr T2
    TGroup  by 5 years
\end{verbatim}
\end{small}

Assuming that T1 and T2 have 1-year resolution, \insql{TGroup} will
produce 20\% fewer graphs to join.  This reordering cannot be applied
every time, however, because an invalid result can be produced.  For
example, if T1 and T2 are those in Figures~\ref{fig:tg}
and~\ref{fig:tg_t2}, then an aggregation would produce graphs with
temporal schema ([2010, 2015), [2015, 2020)) for T1 and ([2008, 2013),
      [2013, 2018)) for T2, which are not union-compatible.  We check
        that the temporal schema of aggregated graphs is
        union-compatible and apply the reordering only if it is.

{\bf \insql{TAnd} join before temporal selection.}  Temporal
intersection between two graphs with a small temporal overlap can cut
down on graph loading time significantly.  Consider this query:

\begin{small}
\begin{verbatim}
   TSelect All V; All E
   From    T1 TAnd T2
\end{verbatim}
\end{small}

There is an implicit temporal selection here over the entire T1 and T2
duration.  If the T1 and T2 temporal bounds are known, then the query
can be rewritten with an explicit temporal selection matching the
length of the overlap.  In the case of example graphs T1 and T2 from
Section~\ref{sec:example}, we can rewrite the query like this:

\begin{small}
\begin{verbatim}
   TSelct All V; All E
   From   T1 TAnd T2
   TWhere T1.Start >= 2010 And T2.End <= 2014 
          And T2.Start >= 2010 And T2.End <= 2014
\end{verbatim}
\end{small}

This rewriting reduces the loading size of T1 and T2 by 30\% and,
since loading time is linearly dependent on loading size (as we show
in the next section), reduces the overall query time by about the same
amount.

{\bf Projection before aggregation or join.}  Recollect that
aggregation and joins operate not only on the graph structure, but
also graph attributes.  In some cases, the values of some or all of
those attributes are irrelevant to the query, and the attribute
aggregation can be greatly reduced or eliminated.  Consider query Q6
from Section~\ref{sec:example}, reproduced here for convenience:

\begin{small}
\begin{verbatim}
     TSelect   All V[vid, pagerank() as pr]; 
               Any E[vid1, vid2, sum(cnt)]
     From      T1 TOr T2
\end{verbatim}
\end{small}

T1 and T2 vertices have two attributes: name and salary, but they are
irrelevant to this particular query.  We can project T1 and T2 to
\insql{V [vid]} prior to evaluation of the \insql{TOr} operation for
better performance.  However, we cannot apply analytics in projection
before the other operations, so we cannot always perform projection
first.

{\bf Data Structure selection.} In addition to selecting the most
efficient order of operations, we select the best data structure for
the query.  Our analysis of data structures' effectiveness for each
individual operation is shown in the next section.  Switching between
data structures from operation to operation is possible but
computationally inefficient.  In general, however, SnapshotGraph
performs better on graphs with a small temporal span, while OneGraph
on those with large span.

\eat{Instead, we pick the structure that
provides the best compromise for the query operations by associating
costs with each operation and picking the least expensive one.}

{\bf Partitioning.} In our experience, the partitioning of the data
affects the performance more than the data structure choice itself.  A
poor partition strategy, in terms of the correct number of partitions
and the allocation of edges to partitions, can be 10x times slower
than a good one, as we show in the next section.  We evaluated
different strategies and partition numbers experimentally and this
serves as a basis of simple rules for partition selection.  For
example, EdgePartition2D strategy provides a significantly better
performance for snapshot-based structural analytics such as pagerank.

Other criteria that can affect the query performance are data skew,
change rate between snapshots, and data size.  We continue to evaluate
these aspects and will incorporate them into the query optimization.


\subsection{Data Representation}
\label{sec:sys:datastructs}

We developed several in-memory representations of evolving graphs to
explore the trade-offs of compactness, parallelism, and support of
different query operators.

{\bf SnapshotGraph (SG).} The simplest way to represent an evolving
graph is by representing each snapshot individually, a direct
translation of our logical data model.  We call this data structure
SnapshotGraph, or SG for short. An example of an SG is depicted in
Figure~\ref{fig:sgp}.  SG is a collection of snapshots, where vertices
and edges store the attribute values for the specific time interval.
A \insql{TSelect} operation on this representation is a slice of the
snapshot sequence, while \insql{TGroup} and temporal joins
(\insql{TAnd} and \insql{TOr}) require a group by key within each
aggregate set of vertices and edges.

While the SG representation is simple, it is not compact, considering
that in many real-world evolving graphs there is a 80\% or larger
similarity between consecutive
snapshots~\cite{DBLP:journals/tos/MiaoHLWYZPCC15}.  In a distributed
architecture, however, this data structure provides some benefits as
operations on it can be easily parallelized, by assigning different
snapshots to different workers, or by partitioning a snapshot across
workers.  

{\bf MultiGraph (MG).}  To take advantage of high similarity between
snapshots, we developed another data structure called MultiGraph, or
MG for short (Figure~\ref{fig:mg}).  MG stores the evolving graph as a
single graph, with one vertex for all time periods, but with one edge
per period where it exists.  Because our goal is to represent both
topological and attribute information, we need to store not only
vertex presence or absence (which can be easily accomplished by an
existence string, like in~\cite{Kan2009}, or by bit sets), but also
the values of vertex attributes at each time period.  MG vertex
attribute, thus, is a map of time indices that represent intervals to
corresponding values.  Edge attributes are tuples of the time index
and the corresponding value.  Vertices typically change infrequently,
so storing each vertex only once reduces the total number of vertices
by about 80\% in our data.  Some of these savings, however, are taken
up by the storage of a more complex map data structure for attribute
values, compared to a single attribute as in SG.  

The implementation of some of the \ql operations in MG is more complex
than in SG.  \insql{TSelect} is a subgraph operation that operates on
all vertices and edges.  \insql{TGroup} on vertices is a combination
of transform and filter operations, since the vertices are already
aggregated across the whole time period, but the edges, like in SG,
are grouped by key within their respective sets.  Implementation of
snapshot analytics like PageRank is done in batch mode, similar
to~\cite{DBLP:journals/tos/MiaoHLWYZPCC15}, by computing the values
and sending messages between vertices for all time periods at once.

Note that for a large subset of the queries, attribute information is
not used, and only the topology is important.  Thus, we can store
vertex attributes in a separate collection (column store), removing
the attribute map and replacing it with existence bit sets instead.
This is the essence of the special case of MultiGraph, called {\bf
  MultiGraphColumn (MGC)}, depicted in Figure~\ref{fig:mgc}.  The MGC
representation allows storage of an arbitrary number of vertex
attributes without using complex per-vertex lists, read from disk only
as needed, and so amenable to lazy evaluation, an important
performance optimization in Apache Spark.  Further compression can be
achieved by storing vertex attribute values only once across all time
periods in which they are the same, similar to how temporal databases
represent this type of data (e.g., see~\cite{Muller2008}).  The
drawback of this approach is that decompression is required to
support, for example, the \insql{TGroup} operation.

{\bf OneGraph (OG).}  The most topologically compact representation is
to store each vertex {\em and} each edge only once for the whole
evolving graph, by taking a union of the snapshot vertex and edge
sets.  The OneGraph data structure, or OG for short, uses this
representation in our system.  Similar to MG, vertex and edge
attributes are stored in time-indexed maps (Figure~\ref{fig:og}).
Compared to MG, this leads to storing about 75\% fewer edges.  The OG
data structure provides some benefits in addition to compactness,
since it reduces the total communication between vertices in
Pregel-based analytics in batch mode.  The drawback is that OG is much
denser than individual snapshots.  As with MG, \insql{TSelect} is a
subgraph operation, and \insql{TGroup} is a transform and filter
operation --- for both vertices and edges.

Similar to MGC, {\bf OneGraphColumn (OGC)} uses a single graph to
represent the union of vertices and edges, with bit sets for presence
information, while attribute information is stored separately.  This
is not as compact as storing attributes within the graph elements, but
is faster in many operations where only graph topology is required.

\begin{figure*}
\begin{minipage}{2.3in}
\centering
\includegraphics[width=2.3in]{figs/E2D_1.pdf}
\caption{E2D with 4 partitions.}
\label{fig:e2d}
\end{minipage}
\begin{minipage}{2.3in}
\centering
\includegraphics[width=2.3in]{figs/Consecutive_1.pdf}
\caption{Consecutive with 4 partitions, 3 snapshots.}
\label{fig:consecutive}
\end{minipage}
\begin{minipage}{2.2in}
\centering
\includegraphics[width=2.2in]{figs/Hybrid2D_1.pdf}
\caption{E2D-Temporal with 4 partitions, 4 snapshots, 2 runs.}
\label{fig:hybrid2d}
\end{minipage}
\end{figure*}

\subsection{Partitioning Strategies}  
\label{sec:sys:partition}

Graph partitioning can have a tremendous impact on system performance.
A good partitioning strategy needs to (1) be balanced, assigning an
approximately equal number of units to each partition, and (2) limit
the number of cuts across partitions, to reduce cross-partition
communication.  

There are two basic types of graph partitioning strategies. Vertex-cut
(also known as edge partitioning) distributes edges across the
available machines and replicates vertices as necessary, while
edge-cut (vertex partitioning) does the opposite.  Vertex-cut
approaches have been shown to have better
performance~\cite{Gonzalez2012}, and are the strategies of choice in
Graph X.  In \ql, we support six different vertex-cut strategies,
which are applied prior to the operation but after loading, and can be
re-applied at any point.

{\bf Canonical Random Vertex Cut (CRVC).}  The source and destination
ids of a vertex are hashed in a canonical direction and the result is
distributed among the available partitions.  The result is a random
vertex cut that co-locates all edges connecting a given pair of
vertices, regardless of direction.  This strategy is available in
GraphX and was used without modification.

{\bf 2D Edge (E2D).}  A sparse edge adjacency matrix is partitioned in
two dimensions (Figure~\ref{fig:e2d}), guaranteeing a $2 \sqrt{n}$
bound on vertex replication, where $n$ is the number of
partitions. E2D can provide good performance for Pregel-style
analytics.  This strategy is available in GraphX and was used without
modification.

{\bf Naive Temporal.} Let $n$ be the number of snapshots and $p$ be
the number of partitions.  If $n \geq p$, then snapshot $i$ is placed
into partition $i \% p$.  Otherwise, multiple partitions are grouped
into a {\em run}, and the CRVC strategy is used to partition a
snapshot within each run.

{\bf Consecutive Temporal.}  Let $n$ be the number of snapshots and
$p$ be the number of partitions.  If $n \geq p$, then temporally
consecutive snapshots are assigned to the same partition.  Otherwise,
each snapshot is assigned an equal number of consecutive partitions,
and CRVC is used to partition the snapshot
(Figure~\ref{fig:consecutive}).

Many networks exhibit strong temporal skew, with later snapshots being
significantly larger than earlier ones.  Consecutive temporal strategy
may result in an unbalanced partitioning for networks with large skew.

Furthermore, the number of cuts for a vertex is the number of
intervals in which it exists with degree > 0.  In the worst case, a
vertex will be cut $n$ times.  Therefore, if vertices persist across
many time intervals, this strategy is not a good choice.

{\bf Hybrid strategies.} Hybrid strategies combine elements of
temporal and structural criteria.  We define two such strategies,
E2D-Temporal (Figure~\ref{fig:hybrid2d}) and CRVC-Temporal.  With both
strategies, we create runs of temporally consecutive snapshots, assign
an equal number of partitions to each run, and then use E2D or CRVC
within each run.

These strategies are specifically designed for time-based operations
such as aggregation to co-locate edges that are to be grouped.  Run
width is determined based on the operation.  For \insql{TGroup}, we
take the number of snapshots in a group as the run width.  For other
operations the default run width is 2.

Note that not all partitioning strategies can be applied to every data
structure.  For example, only purely structural strategies can be
applied to OG.  We report on the experimental effectiveness of the
strategies in Section~\ref{sec:exp}.


