\section{Introduction}
\label{sec:intro}

The development of SQL, a declarative query language for relational
data analysis, had a tremendous impact on the usability of database
technology, leading to its wide-spread adoption. At the same time, the
separation between the logical and the physical representations paved
the way for powerful performance optimizations. The motivation for our
research is to provide a similar tool for analyzing evolving graphs,
an area of interest in many research communities, including sociology,
epidemiology, networking, etc.



% General motivation
\eat{
The importance of networks in scientific and commercial domains cannot
be overstated.  Considerable research and engineering effort is being
devoted to developing effective and efficient graph representations
and analytics.  Efficient graph abstractions and analytics for {\em
  static graphs} are available to researchers and practitioners in
scope of open source platforms such as Apache Giraph, Apache Spark /
GraphX~\cite{DBLP:conf/osdi/GonzalezXDCFS14} and GraphLab /
PowerGraph~\cite{DBLP:conf/osdi/GonzalezLGBG12}.
}

\eat{
Arguably the most interesting and important questions one can ask
about networks have to do with their evolution, rather than with their
static state.}

Analysis of {\em evolving graphs} has been receiving increasing
attention~\cite{DBLP:journals/csur/AggarwalS14,Miao2015,Ren2011,Semertzidis2015}.
Yet, despite much recent activity, and despite increased variety and
availability of evolving graph data, systematic support for scalable
querying and analysis of evolving graphs still lacks.  This support is
urgently needed, due first and foremost to the scalability and
efficiency challenges inherent in evolving graph analysis, but also to
considerations of usability and ease of dissemination.  The goal of
our work is to fill this gap.  In this demonstration we present
Portal, an open-source distributed framework that implements an
algebraic query language called \tg algebra~\cite{PortalarXiv2016}
(\tga) and a declarative language \tgql over it. Portal streamlines
exploratory analysis of evolving graphs, making it efficient and
usable, and providing critical tools to computational and data
scientists.

\julia{What do we want to say here?  It's confusing to be zooming in
  on \tga, let's rephrase to talk about (a) why a relational treatment
  of evolving graphs and (b) why Spark.}  Our goal in developing \tga
is to give users an ability to concisely express a wide range of
common analysis tasks over evolving graphs, while at the same time
preserving inter-operability with temporal relational algebra (\tra).
Implementing (non-temporal) graph querying and analytics in an RDBMS
has been receiving renewed
attention~\cite{DBLP:conf/sigmod/AbergerTOR16,DBLP:conf/sigmod/SunFSKHX15,DBLP:journals/pvldb/Xirogiannopoulos15},
and our work is in-line with this trend. Our data model is based on
the temporal relational model, and our algebra corresponds to temporal
relational algebra, but is designed specifically for evolving graphs.

In what follows, we briefly describe our data model and algebra (Section 2), declarative query
language for evolving graphs (Section 3) and the Portal system
(Section 4) that implements it. We then present several
interesting demonstration scenarios (Section 5). We finish
with a brief discussion of related work (Section 6) and with
take-home messages (Section 7).


