\section{Temporal Graph Model}
\label{sec:tga}

\begin{figure}[t]
\centering
\includegraphics[width=3in]{figs/T1_rel.pdf}
\caption{\tg \insql{T1}.}
\label{fig:tg_rel}
\end{figure}

Our data model is based on the temporal relational model, and our
algebra corresponds to temporal relational algebra, but is designed
specifically for evolving graphs.  

{\bf Data model.}  In~\cite{PortalarXiv2016} we proposed an evolving
graph model \tg based on temporal relations with point semantics.
Briefly, an evolving graph consists of two temporal relations,
\insql{V} and \insql{E}, that represent the vertices and edges of the
graph with their corresponding periods of validity expressed by
intervals.  Optionally, two other relations \insql{VA} and \insql{EA}
represent the vertex and edge attributes using the property model,
also with their periods of validity.  An example is shown in
Figure~\ref{fig:tg_rel}.

A \tg represents a single graph, and models evolution of its topology
and of vertex and edge attributes.  A snapshot of the graph is the
state of the relations at any time point.  Relations of \tg are
coalesced~\cite{DBLP:conf/vldb/BohlenSS96} --- each fact (existence of
a vertex or edge, or an assignment of a value to a vertex or edge
attribute) is represented exactly once for each time period of maximal
length during which it holds.  Referential integrity holds on
\insql{E} w.r.t. \insql{V}, guaranteeing that edges only exist if
their end points exist at the same time, on \insql{VA}
w.r.t. \insql{V}, and on \insql{EA} w.r.t. \insql{E}.

{\bf Operations.} In~\cite{PortalarXiv2016} we proposed a temporal
graph algebra \tga.  The algebra is compositional: operators take a
\tg or a pair of \tgs as input, and output a \tg.  \tga semantics can
be expressed as a sequence of temporal relational algebra expressions,
guaranteeing snapshot reducibility and extended snapshot
reducibility~\cite{DBLP:reference/db/Bohlen092} --- two properties
that are appropriate for a point-based temporal data model.  The
operations of \tga include slice, map, selection, aggregation, node
creation, edge creation, and common binary operations union,
intersection, and difference.  We study the expressiveness of \tga
in~\cite{PortalarXiv2016} and show completeness w.r.t. temporal
relational algebra.  We also support Pregel-style analytics that are
logically executed on each snapshot.

In the next section, we explain the operations of TGA through
examples, and refer the reader to~\cite{PortalarXiv2016} for
additional details.  The next section also serves to introduce the
syntax of \ql, a declarative language, whose queries are rewritten by
the \sys system into sequence of \tga operations.  \ql is being
presented here for the first time.
