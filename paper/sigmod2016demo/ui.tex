\section{Graphical Query Composition}
\label{sec:gui}

\begin{figure}
\begin{center}
\includegraphics[width=3.2in]{figs/ui.pdf}
\caption{Graphical query composition. PLACEHOLDER}
\label{fig:ui}
\end{center}
\end{figure}

Evolving graph analysis is of interest to researchers in many domains,
as we already showed previously.  While \ql is declarative and, thus,
does not require computer science expertise to use, we want to reach a
wide audience.  For this reason, we are developing a graphical query
composition tool \qlui (Figure~\ref{fig:ui}).

\qlui users can compose queries by adding to and manipulating
\tg\\s in the workspace, where they are represented by their
timelines.  To temporally join two evolving graphs, for example, the
user can drag them to the workspace and place them close together.  If
the two graphs are not union-compatible, they will not snap together
and an error message is displayed to the user explaning the problem.
Union-compatibility is one of the more complex aspects of the
language, and it is our hope that a graphical representation of the
\tg temporal schema will reduce confusion.

To select a temporal subset of a \tg, the user can move the outside
border of the timeline.  To aggregate a \tg, the user can move the
border of any of the timeline periods and then specify the aggregation
functions through a drop-down menu.

As the query is composed graphically, its \ql representation is
updated in the Query view.  The user can also view the optimized
logical plan for the query in the Plan view.
