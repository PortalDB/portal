\section{Introduction}
\label{sec:intro}

Motivation is to enable efficient computation on large evolving graphs
in a distributed environment.  Previous work on temporal vs structural
locality was carried out in a centralized setting (ImmortalGraph).
While those lessons are useful as a starting point, they do not always
apply to a dstributed environment.  Miao do describe some limited
experiments in a distributed environment, which should be taken with a
large grain of salt because their experimental setup was not
representative of typical clusters -- 4 multicore servers, each with
16 cores and 128 GB of memory) connected by InfiniBand (40 Gbps).
Even in this setup they showed only small gains of their approach
compared to executing snapshot by snapshot.

We have several options for distributed computation: partition
temporally, partition spatially, or some combination of the two.
Different partition strategies enable different method of computation.
Let's use snapshot analytics as our running example, as they are the
most computation and communication intensive of all operations.

{\bf Temporal partitioning.}  We can allocate some number of snapshots
to each partition, depending on the cluster configuration.  This
assumes that one or more snapshots fit into memory of the partition.
